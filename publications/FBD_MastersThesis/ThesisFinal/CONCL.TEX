\chapter{Conclusions and Recommendations}

	\section{Conclusions}

	A major problem in CAD/CAM is to device appropriate representations for
	modeling product information in a computer. Feature-based design offers a
	tool for mapping a designer's abstraction into a representational form
	that is suitable for manufacturing and other downstream applications.

	In the research work presented herein, we were able to successfully
	develop and implement a feature-based design system for generalized
	orthohedral components.

	A data representation scheme was developed for modeling three
	dimensional solid components with orthohedral external shape and a wide
	variety of internal shapes. 
	The geometry and topology are explicitly represented.

	The feature library is extendable and customized to suit the needs of 
	any given application. This extendability is also exploited to provide 
	support for free-form 
	surfaces ( extruded B-Spline ) through the mechanism of encapsulation.
	This capability offers flexibility in enriching the feature set and
	supporting customized features.


    The family of features currently supported includes :
        \begin{itemize}
        \item
        Constructive Block
        \item
        Subtractive Block
        \item
        Constructive Cylinder
        \item
        Subtractive Cylinder
        \item
        Constructive Wedge
        \item
        Subtractive Wedge
        \item
        Constructive swept B-Spline surface
        \end{itemize}


	Operations such as {\em Create}, {\em Edit}, {\em Display}, {\em Save} and
	{\em Retrieve} are also supported.

	The architecture of the prototype system is {\em open}, providing the
	infrastructure for additional development to enhance its functionality and 
	modeling capabilities.

	Feature-based design systems are generally lacking in
    dimensioning and tolerancing support. The present work develops and
    implements a robust dimensioning and tolerancing model with the capability
	to  create, modify and validate part dimensioning schemes.


	C++ is used to implement the proposed feature-based design concepts
	using the object-oriented methodology. 
	The user-interface is developed in the X-Windows Motif toolkit, while the
	graphics display is done using PHIGS.
	The user-interface and application classes are separated as much as
	possible, thereby providing the capability to change the user interface
	with minimal modifications in the object classes themselves.


    \section{Recommendations}

	The FBD prototype developed can be used to model components with
	orthohedral, cylindrical or prismatic geometry, with limited support
	for free-form surfaces. This limitation can be reduced by adding new
	geometric features to the available set of geometric features. The
	mechanism of encapsulation could be used to model much wider classes of
	geometric shapes and topological links.

	Since the prototype FBD system supports both design and manufacturing 
	features, it could be extended to generate process planning data.

	The D \& T facility that is provided has limited support for tolerance 
	analysis. Useful work could be done in this area to provide the capability 
	for tolerance synthesis and validation. The user interface for the D \& T 
	portion of the software has a lot of room for improvement.

	In the present prototype FBD package, the features are not written
	in internationally recognized formats. This capability would allow
	other applications such as finite element analysis or NC code generation
	to use the current solid modeling system for modeling the components.
	Thus, support for international standards such as PDES/STEP and industry
	standards like DXF will greatly improve the usability of the software.

