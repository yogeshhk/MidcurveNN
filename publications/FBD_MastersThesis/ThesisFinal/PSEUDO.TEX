    \subsubsection{Creation Process }
 
    \paragraph{New Component}
 
        Creation of a component is started by the creation of the first block
    purely from geometrical data. This block serves as the base to which other
    blocks can be added to form the desired component.
 
 
    \begin{verbatim}
 
    *   call back for New Component
        Make_data_entry_table()
 
    OK =    "data_entry_call back.c"
        Get data values from widget
        new Component {
                new Block
                Link_blocks
                new Component Graph(comp,block){
                    new Feature node
                    Link_Nodes()
 
                }
 
        }
 
        Link_comp();Set currc
        comp_flag = OLD_COMP;
        Display_by_phigs ( new comp);
 
 
 
    CANCEL = "data_entry_call back.c"
 
        reinitialize data_flags and other flags appropriately
 
    \end{verbatim}
 
 
    \paragraph{Adding Block to Existing Component}
 
    To add a block to the existing component , the type of topology that will
    exist between the new block and the preexisting block needs to be specified
    Depending on the type of the topological link, certain planes of the new
    block get fixed according to the dimensions of the planes of the old block.
 
 
    \begin{verbatim}
 
    *   call back for any addblk option in Create
        Set block_flag to the appropriate block_type
 
    *   Ask for number of topological links
 
    *   FOR those many times create Topo_Selection Box +
        final Full_data_entry_table( " topo_selection_callbacks.c")
 
    *   In Topo_selection_box
            Select Parent_Block ( set blk_num)
            Select topological link ( Set topo_type)
            Get reqd planes and store them in array of Planes[6]
 
        APPLY = ( for every Topo_selection_Box )
 
            Fill the latest ( globally) available information in
            array of Topo_info[10]
 
            switch( topo_type) fill up data_flags //
                // in "topo_data_assignments.c "
 
            Write_in _data_table( data_flags);
 
            OK = ( data_table )("data_entry_callbacks.c")
 
                Get values from data widgets
                Depending on block_type instantiate Block
                Link_Blocks to currc
 
                new Feature Node( new block)
                // sets it in
 
                comp_graph->Link_nodes // Vertically
 
                We still have information stored in Topo_info[i]
 
 
                For each topo_info[i] depending on the topo_type
                instantiate new Topo link(topo_info,new blk) {
                    new Feature Node
                    set new blk in it
                    new arcs
                    put Topo links in them
 
                    Join nodes by arcs
                    Join nodes in everyone s node list
                    Join arcs in everyone s arc list
                }
 
                data flags are reinitialized
 
            CANCEL =
 
                initialize Planes[6]
                initialize topo_info[10]
                initialize data_flags
 
    \end{verbatim}

%%%%%%%%%%%%%%%%%%%%%%%%%%%%%%%%%%%%%%%%%%%%%%%%%%%%%%%%%%%%%%%%%%%%%%%%%%

 
    \subsubsection{Complete Procedure}
 
    \begin{itemize}
 
    \item
   Set current component
    \item
   Select Block ( Node ) for editing
    i.e length to be edited and the constraint plane
 
    \item
   FINDCYCLES for all the nodes ( actually for the time being I am
 
    doing it for the edited node only to save the extra computations
 
    \item
   Process the cycles to remove all the cycles having path of 2 nodes
    only and removing the duplicate Cycles
 
    \item
   Present the cycles to the user to pick for processing
 
    \item
   If there are NO cycles then he indicates such thing or else if he
    he wants cycle processing then he has to give the modification node
 
    \item
   while traversing the cycle from both sides go on putting the  nodes
    in the queue (respective to the axis chosen)
 
    \item
   If you dont want cycles then you come to following steps directly
 
    \item
   Put this node in the Queue ( of the type of the length to be edited)
 
    \item
 
    \begin{verbatim}
 
    for( XQ, YQ, ZQ){
 
        while( not end of Q){
 
            Set current node as Base Node
 
            Modifylength();
 
            // Searches for "lengthid" in the LengthListid and
            // finds other links.
            // if edited length is different and NOT fixed Add
            // those other nodes to the Qs of that Length type
 
            if all the adjacent nodes of Base Node are Resolved
 
                mark the base node as VISITED
                Remove this Base node from the Q
 
            Go to the Next Node in the Q
 
        }
 
    }// end of all Qs
 
    \end{verbatim}
 
    \item
   Done.
 
	\end{itemize}
%%%%%%%%%%%%%%%%%%%%%%%%%%%%%%%%%%%%%%%%%%%%%%%%%%%%%%%%%%%%%%%%%%%%%%%

    The whole process is explained in the following algorithm
 
 
    \begin{verbatim}
 
    Go to each block{
 
        Create_Faces();
        Make_them_polygons(Face*);  // within polygon, calculate Normal &
                                    // distance from origin
 
        Traverse_List_of_polygons(){
 
            if(normal matches )Add_to_existing list of Polygonset;
            else{
 
                Create_new_polygonset and add the polygon to it;
                Add polygon set to the list of polygonsets in curr_component;
            }
 
        }
 
    }
 
    Go_to_each polygonset{
 
        See if any of the two have common edge;
        if(yes){
                //Merge the polygons to first and delete the second one
 
                Merge(){
 
                    // Take two polygons from the polygonset P1,P2
                    Join_if_edge_common(P1,P2){
 
                        // Go to each pair of vertices in P1 and check
                        // with each vertex of P2
 
                        check = Test_edges(V1,V2,W1,W2);
                        if(check){
 
                            // Collect vertices in P1
                    }
                }
                Test the Merged polygon to further check the common edges
                and split it if necessary
        }
    }
 
    Go to each polygon set and Display_by_phigs();
 
    \end{verbatim}

