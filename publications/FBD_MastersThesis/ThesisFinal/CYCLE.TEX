    \chapter{Cycle detection in Graphs} \label{cyldetect}

    \section{Definitions}

    \begin{itemize}

    \item
    An undirected graph G is {\em connected} if there is at least one
    path between every pair of vertices $v_{i}$ and $v_{j}$ in G.
    \item
    A directed graph G is connected if the undirected graph obtained by
    ignoring the edge directions in G is connected.
    \item
    A directed graph is said to be strongly connected if for every pair of
    vertices $v_{i}$ and $v_{j}$ there exits at least one directed path from
    $ v_{j} $to $v_{i}$.

    \end{itemize}

	\section{The algorithm}

    The simplest problem, of course, is to determine whether a given graph,
    directed or undirected contains a cycle. Depth first search suffices
    to answer the question for both directed and undirected graphs : a graph
    or a digraph has a cycle if and only if the DFS tree contains a back edge.
    A slightly harder problem is to determine a minimal set of cycles of an
    undirected graph from which all of the cycles in the graph can be
    constructed ( a so called fundamental set of cycles).It is still harder
    to determine all the cycles of a digraph. ~\cite{Rein}
    With the combination of the fundamental cycles all other cycles can be
    derived.
        At first a spanning tree of the graph is generated with Depth first
    search. When algorithm finds a back edge ( an edge is called Back edge
    because they lead back to vertices that have been previously visited),
    the cycle consists of the tree edges from the current node to the
    destination node of the edge and the current edge. So as to collect proper
    edges of tree a stack is used while performing depth first search which
    stores all the vertices on the path from the root to the vertex currently
    being visited. When the back edge is encountered, the cycle formed
    consists of that back edge and the edges connecting the vertices on the
    top of the stack.


        Any algorithm for generating all the cycles of a
    digraph can also be used for undirected graphs by converting the undirected
    graph into a digraph by replacing each edge with two oppositely directed
    edges between the same pair of vertices.~\cite{Rein}


        The graph structure used here is actually bidirectional with directed
    edges in both directions. Precaution is taken to remove all the cycles with
    two edges and  thereby avoiding cycles between two adjacent nodes.


        Extra care needs to be taken so that every cycle detected is new one and
    not just combination of the already found ones.The process employed in
    detection of cycles is based on CYCLE algorithm ~\cite{Rein} and
    is explained briefly here.


    All the nodes are numbered in an ascending order. This is done at the
    creation stage only. We begin the search at a vertex $s$ and build a
    (directed) path $(s,v_{1},v_{2},v_{3}....,v_{k})$, such that
    $ v_{i} > s, 1 \leq i \leq k$.
    A cycle is reported only when the next vertex $ v_{k+1} = s $.
    After generating the cycle $ (s, v_{1},v_{2},... v_{k},s) $,
    we explore the next edge going out of $v_{k}$. If all the edges going out
    of $v_{k}$ have been explored, we back up to the previous vertex
    $ v_{k-1}$ and try extending from it and so on. This process continues
    until we finally try back up past the starting vertex $ s $;
    all the cycles containing vertex $ s$ have been generated. This process is
    repeated for all the nodes if we want to find out all the cycles. In the
    current implementation the cycles having the edited node are only presented
    to user for selection.


    All the vertices are made {\em unavailable} ( by setting {\em available }
    flag in the node as FALSE as soon as $ v $ is appended on the current path)
    while searching with graph rooted at $ s $ so as to avoid the traversing of
    the cycles originating at $ v_{i}$.It remains {\em unavailable} until search
    backs up past $ v $ to the previous vertex on the current path. A record is
    kept of the predecessors of all unavailable vertices that are not on the
    current path by maintaining a separate set for each vertex. When $ u $

    becomes available so do all its unavailable predecessors that are not in
    the current path. Thus all the elements in that set are made available. As
    these members are made available again, their unavailable predecessors that
    are not the current path are made available again and so on.This recursive
    procedure is called UNMARK in the implementation.


    Every cycle in a digraph lies entirely within a strongly connected
    component. So at start graph is decomposed into strongly connected
    components.


    After finding strong components, the above mentioned procedure is applied
    to every component at a time. In the current implementation, cycle
    detection is done only for the component that contains the edited node.


