\thispagestyle{empty}
\begin{center}
{\huge \bf {Abstract}}
\end{center}

\vspace{0.2in}

    Increased demands on productivity in manufacturing have lead to increased
	automation of design and manufacturing processes.
    Automating CAD and CAM independently does not necessarily provide the 
	``best'' solution for the overall product realization process.
	Computer integration of both these activities is the potential solution.

    This thesis proposes a feature-based design approach for three
    dimensional block structured components with generalized orthohedral
	geometry. 
	The component is viewed as a collection of blocks connected by 
	topological links. The geometry and topology are explicitly defined.
	The geometry of the feature consists of an external shape and an internal
	shape, while the topology is modeled using a graph. The basic external 
	shape of each block in a component is a rectangular parallelopiped which
	fixes overall size and location of the block.
	Using the idea of encapsulation, families of blocks with different internal 	shapes can be developed; all these internal shapes are assumed to be
	encapsulated in a block with rectangular external shape. The simple 
	nature
	of the external shapes makes it easier to define operations like create, 
	addition/deletion of blocks, topological connectivity specifications, etc.
	The internal shape provides the necessary information for geometry specific
	operations such as display, volume computation, etc.

    The feature library consists of constructive as well as subtractive 
	features. The modeling of free-form surfaces using the
	idea of encapsulation is proposed and an extruded B-Spline 
	surface is developed and implemented as an example.


	An editing method has been developed to maintain topological consistency 
	during geometrical modifications of the component and to propagate
	the editing changes with minimum user intervention.


	Dimensioning and tolerancing support is provided using a graph
	representation. Capabilities
	such as automatic creation of a dimensioning scheme, change of datum,
	detection of over-dimensioning and under-dimensioning, etc are provided.

    C++ is used to obtain an object-oriented implementation of the feature-
	based design concepts developed in this work.
    The user-interface is developed in X-Windows/Motif toolkit while the
    graphics display is done using PHIGS.


