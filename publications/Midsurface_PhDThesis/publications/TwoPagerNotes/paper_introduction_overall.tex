\section{Introduction}\label{sec:intro}

% Getting a quicker validation of the proposed product is crucial in the era of fierce competition and faster obsolescence. Digital product development, which includes, modeling by CAD (Comouter-aided Design)  and analysis by CAE (Computer-aided Engineering) plays a crucial role in quicker ``Time to market''.  For thin-walled models such as sheet-metal/plastic products, a quicker and fairly accurate CAE analysis is possible by idealizing them to their equivalent surface representations, called ``Midsurface''. Midsurface can be envisaged as a surface lying midway of a thin-walled solid, mimicking its shape.   In CAE analysis, instead of using expensive 3D solid elements, 2D surface elements are used on the midsurface for fairly accurate results in lesser computations/time.  Even in the age of scalable and near-infinite computing power, it is still desirable to have  a robust, well-connected midsurface, so as to be able to run more design iterations, quickly.  Because of such advantages, the midsurface functionality is widely used and is available in many commercial CAD-CAE packages. 

In the era of intense competition and rapid product obsolescence, accelerating the validation of proposed products is crucial. Digital product development, encompassing computer-aided design (CAD) modeling and computer-aided engineering (CAE) analysis, plays a pivotal role in reducing time-to-market. For thin-walled models, such as sheet metal or plastic products, a quicker and reasonably accurate CAE analysis can be achieved by idealizing them as their equivalent surface representations, known as "midsurfaces." A midsurface can be envisioned as a surface lying midway through a thin-walled solid, mimicking its shape. In CAE analysis, instead of employing computationally expensive 3D solid elements, 2D surface elements are utilized on the midsurface, yielding fairly accurate results with reduced computational time and resources. Even in the age of scalable and near-infinite computing power, a robust, well-connected midsurface remains desirable to facilitate iterative design exploration rapidly. Due to such advantages, midsurface functionality is widely utilized and available in numerous commercial CAD-CAE packages.


\begin{figure}[h]
% \begin{minipage}[h]{\linewidth} 
% \begin{minipage}[h]{0.5\linewidth} 
		\centering
		\includegraphics[width=0.9\linewidth]{images/MidsurfaceErrorsMscApex} % 
		\captionof{figure}{Midsurface Errors (Source: \cite{MScApex})}
		\label{fig:midsurfaceerrors}
% \end{minipage}
% \hfill
% \begin{minipage}[h]{0.5\linewidth} 
% In spite of its demand and popularity, the existing techniques of computing the midsurface fail to compute a well-connected midsurface, especially for non-trivial shapes (\cite{Woo2013,Automex}). Failures manifest in the form of gaps, missing patches, overlapping surfaces, not lying midway, not mimicking the input shape, etc. (Figure \ref{fig:midsurfaceerrors}). Correcting these errors is mostly a manual, tedious and highly time-consuming task, requiring hours to days. This correction time can be nearly equivalent to the time it can take to create the midsurface manually from scratch (\cite{Stolt2006}). 
% \end{minipage}
% \end{minipage}
\end{figure}

% In spite of its demand and popularity, the existing techniques of computing the midsurface fail to compute a well-connected midsurface, especially for non-trivial shapes (\cite{Woo2013,Automex}). Failures manifest in the form of gaps, missing patches, overlapping surfaces, not lying midway, not mimicking the input shape, etc. (Figure \ref{fig:midsurfaceerrors}). Correcting these errors is mostly a manual, tedious and highly time-consuming task, requiring hours to days. This correction time can be nearly equivalent to the time it can take to create the midsurface manually from scratch (\cite{Stolt2006}). 

Despite its demand and popularity, existing techniques for computing midsurfaces often fail to generate well-connected midsurfaces, particularly for non-trivial shapes (Woo 2013, Automex). These failures manifest as gaps, missing patches, overlapping surfaces, deviations from the midway position, and inaccurate representations of the input shape's geometry (Figure 1). Correcting these errors is predominantly a manual, tedious, and highly time-consuming task, often requiring hours or days of effort. This correction time can be nearly equivalent to the time required to create the midsurface manually from scratch (\cite{Stolt2006}).


 % Automated and  robust technique for computing midsurface  is a crucial need and this work is a step in that direction. Simplification, abstraction and decomposition are the core themes of the proposed approach.
Developing automated and robust techniques for computing midsurfaces is a crucial need, and this work represents a step in that direction. Simplification, abstraction, and decomposition form the core themes of the proposed approach.
