Computer-aided Design (CAD) models of thin-walled parts such as sheet metal or plastics are often reduced dimensionally to their corresponding midsurfaces for quicker and fairly accurate results of Computer-aided Engineering (CAE) analysis. Generation of the midsurface is still a time-consuming and mostly, a manual task due to lack of robust and automated techniques.  Midsurface failures manifest in the form of gaps, overlaps, not-lying-halfway, etc., which can take hours or even days to correct. Most of the existing techniques work on the complex final shape of the model forcing the usage of hard-coded heuristic rules, developed on a case-by-case basis. The research presented here proposes to address these problems by leveraging feature-parameters, made available by the modern feature-based CAD applications, and by effectively leveraging them for sub-processes such as simplification, abstraction and decomposition. 

In the proposed system, at first, features which are not part of the gross shape are removed from the input sheet metal feature-based CAD model. Features of the gross-shape model are then transformed into their corresponding generic feature equivalents, each having a profile and a guide curve. The abstracted model is then decomposed into non-overlapping cellular bodies. The cells are classified into midsurface-patch generating cells, called `solid cells' and patch-connecting cells, called `interface cells'. In solid cells, midsurface patches are generated either by offset or by sweeping the midcurve generated from the owner-feature's profile. Interface cells join all the midsurface patches incident upon them. Output midsurface is then validated for correctness. At the end, real-life parts are used to demonstrate the efficacy of the approach.
