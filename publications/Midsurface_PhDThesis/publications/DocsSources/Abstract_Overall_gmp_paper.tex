%\begin{abstract}


Computer-aided Design (CAD) models of thin-walled solids such as sheet metal/plastic parts are often reduced dimensionally to their corresponding midsurfaces for quicker Computer-aided Engineering (CAE) analysis, while still maintaining fairly accurate results. Computation of the midsurface is still a time-consuming and mostly manual process, due to lack of automated and robust  techniques. Midsurface failures manifest in the form of gaps, missing patches, overlapping surfaces, etc., which take hours or even days to correct. This research proposes the use of feature information for devising a generic logic for the computation of a well-connected midsurface.

In the proposed approach, first, the  irrelevant features are removed from a feature-based CAD model of a thin-walled solid to compute its gross shape. Features then undergo abstraction-transformation to become their generic sweep-based equivalents, each having a profile and a guide curve. Sweep-feature volumes are then decomposed into non-volumetrically-overlapping cellular bodies. A graph is populated with the cellular bodies as nodes and fully-overlapping-surface interfaces as the edges. The nodes are classified into midsurface-patch generating nodes (called `solid cells' or $sCell$s) and interaction-resolving nodes (`interface cells', $iCell$s). In $sCell$, using the owner-sweep-feature information, a midsurface patch is generated by sweeping the midcurve of the feature's profile along with its guide curve. Midsurface patches are then connected in the $iCell$s in a generic manner, thus resulting in a well-connected midsurface with minimum failures. Output midsurface is then validated topologically for correctness.

%{\bf Keywords}: CAD, CAE, FEA, Model Simplification, Midsurface, Feature-based Design
%\end{abstract}