
\section{Related Work}	

One of the early works in the field of skeletonization was by theoretical biologist Harry Blum in1967 \cite{Harry1967} with the medial axis of the shape. Survey papers like \cite{Attali2004}, \cite{Lam1992} and \cite{Yogesh2010} have detailed various approaches used in the skeleton creation. Decomposition of planar shapes  into regular (non-intersecting) and singular (intersecting regions) and its application to skeletonization has been widely researched \cite{Rocha99} as well.

Rocha \cite{Rocha98, Rocha99} used both Decomposition and Skeletonization for character recognition in images. From the results presented it appears that junctions like L, T were far from being well-connected. 

Keil's algorithm \cite{Keil94} finds all possible ways to remove reflexity  of  these  vertices,  and  then  takes  the  one  that  requires  fewest diagonals. 

Lien et. al. \cite{Lien2004} decomposed polygons 'approximately' based on iterative removal of most significant non-convex feature. Other methods of polygon decomposition are based on use of curvature to identify limbs-necks, using Axial Shape Graph, using events occurring during computation of Straight Skeleton, using Delaunay triangulation etc. Many of these methods need quite a bit of pre and post processing to take care of boundary noise \cite{Lien2004}. 

Zou et. al.\cite{Zou2001} partitioned a shape by Delaunay triangulation, identified regular and singular regions and then created skeletons.

Following paper leverages Polygon decomposition algorithm by Bayazit \cite{Bayazit}, suggests improvements in partitioning and then uses it's output for Midcurves creation.

