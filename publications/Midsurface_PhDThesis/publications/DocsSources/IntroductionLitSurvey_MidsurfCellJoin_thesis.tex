%% Introduction 
%%	What is this chapter all about? 
%%	What sub-problem or issue is this chapter addressing? 
%%	How does this chapter fit within the overall “story” of the thesis? 
%%The Meat
%%	Rigorous approach to sub-problem, or detailed explanation of issue
%%	Assumptions underlying sub-problem, or complete description of issue
%%	Validation: System design, theory, implementation, graphs, references, …. 
%%Summary
%%	Repeat the highlights of the chapter
%%	Transition sentence that acts as a “teaser” for the next chapter, and how the next chapter fits with the current one

\section{Introduction}
This chapter presents algorithms for generating a quality midsurface from a generalized feature-based CAD model. 

Despite a wide demand for midusrface, its quality has not improved over time. The literature survey findings, quoted in Chapter~\ref{ch:Survey}, show that the existing approaches fail to compute a well-connected midsurface, especially in the case of complex models. One of the widely used approaches of midsurface generation, the Face Pairing approach, \replaced{has several limitations such as}{, appears limited due to} difficulties in detecting face pairs for computing midsurface patches and devising a generic \deleted{(non-heuristic)} approach for joining these patches. Typical failures are, missing midsurfaces, gaps, overlaps, midsurfaces not lying midway,  etc. Correcting these errors is mostly a manual, laborious and time-consuming process, requiring from several hours to days. This chapter proposes use of feature based cellular decomposition of the generalized feature based CAD model to address these issues \replaced{which are explained in the sections to follow. }{, to device a generic approach for computing a well-connected midsurface.}

