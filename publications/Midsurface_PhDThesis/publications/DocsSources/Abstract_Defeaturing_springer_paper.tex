At the conceptual design phase, relatively complex solid models created by Computer Aided Design (CAD) applications are not sent directly for analysis in Computer Aided Engineering (CAE) applications. They are first simplified so as to save on computational resources and time, without compromising much on accuracy of the results. This process, called Model Simplification, primarily involves task of suppressing small-irrelevant features. This task, called De-featuring, is still largely a manual and time-consuming process owing to the complexity of shapes and due to domain expertize needed. Some attempts to automate the task have been made, but they are far from being practically useful, especially in case of complex shapes.

Many automated simplification techniques use final shape of the CAD model as input, and do not utilize the feature information exposed by many commercial CAD models. Although final act of suppressing or deleting a feature is straight-forward using feature tree, the selection of suppressible features is still a subject of research.

This paper reviews some of the past attempts of de-featuring using feature information and issues-limitations in them. It then presents novel algorithms, both, generic and  specific to Sheet Metal parts.


