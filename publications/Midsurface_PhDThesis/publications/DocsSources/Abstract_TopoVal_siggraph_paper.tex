
Complex models developed in Computer Aided Design (CAD) applications are often simplified before analyzing them in Computer Aided Engineering (CAE) domain, especially during the initial stages of design, where quick (but a bit-approximate) results are acceptable. Thin walled parts, such as Sheet Metal parts are often simplified to a set of surfaces lying midway, called Midsurfaces. They, along with thickness information, are used in creation of shell elements during CAE meshing. 

Midsurface is expected to mimic the shape of the original part, both geometrically and topologically.  Distance from the Midsurface to its corresponding faces in the original part, should be around half of the thickness. Topologically, it should represent all junction-connectivities between sub-shapes, similar to those present in the original part.

This paper attempts to derive relationship between Sheet Metal part (Solid) and its Midsurface (Surface), in both directions (from Solid to Surface and then from Surface to Solid). It goes on to prove the efficacy of the derivation using examples.

