%Paragraph 1: What is the problem? Not more than 3-4 sentences telling the reader what the problem is, in as simple English as possible
%Paragraph 2: Why is the problem hard? What has eluded us in solving it? What does the literature say about this problem?  What are the obstacles/challenges? Why is it non-trivial? 
%Paragraph 3: What is your approach/result to solving this problem?   How come you solved it? Think of this as your “startling” or “sit up and take notice” claims that your thesis will plan to prove/demonstrate 
%Paragraph 4: What is the consequence of your approach? So, now that you’ve made me sit up and take notice, what is the impact? What does your approach/result enable?  


Computer-aided Design (CAD) models of thin-walled solids such as sheet metal or plastic parts are often reduced dimensionally to their corresponding midsurfaces for quicker and fairly accurate results of Computer-aided Engineering (CAE) analysis.  A midsurface is a surface lying midway of (and representing) the input shape.  Computation of the midsurface is still a time-consuming and mostly, a manual task due to lack of robust-automated approaches. Many of the existing automatic midsurface generation approaches result in some kind of failures such as gaps, missing patches, overlapping surfaces, etc. It takes hours or even days to correct such errors with manual intervention. The widely used Boundary Representation (Brep) based approaches are computationally intensive, yet cannot guarantee flawless midsurface and are mostly developed for a limited variety of geometric and topological configurations.  In these approaches no feature information is efficiently leveraged. Thus, there exists a need to take a holistic look at the CAD model with its feature information and devise a set of efficient algorithms to generate flawless, well-connected midsurfaces capable of handling wide variety of geometric and topological configurations and are also computationally inexpensive. %Thus, an automatic, generalized and deterministic algorithm for computation of a well-connected midsurface is the need of the hour.
%Most of the existing approaches work on the final shape (typically in the form of Boundary representation, B-rep). Complex B-reps make it hard to detect sub-shapes for which the midsurface patches are computed and joined, forcing the usage of hard-coded geometric/heuristic rules, developed on a case-to-case basis. The research presented here proposes to address these problems by leveraging feature-information made available from the modern CAD applications. 

This thesis is primarily aimed at addressing this need. It provides an integrated approach to address the critical aspects of generation of midsurface for feature based sheet metal CAD model through design and implementation of an intelligent system, called \mysystemname~({\bf Mid}surface {\bf A}lgorithms for {\bf S}heet-metal-parts). It uses CAD models built using Autodesk Inventor and its Application Programming Interfaces (APIs) are used to interact with the model. The algorithms for various modules are implemented in VB and C\# .Net programming languages.
%
%Features carry a higher level information (such as designer's intent, dependencies, parameters, etc.) than what's available in the solid (Brep). In the past, they were hidden due to proprietary reasons, but now have started becoming available through Application Programming Interfaces (APIs). Feature information can be leveraged effectively in the existing CAD algorithms, where some sort of feature recognition used to happen before proceeding with the core algorithm. In this work, algorithms for model-simplification, decomposition and midsurface generation, use feature information advantageously. 

The thesis begins by providing an overview of CAD-CAE process, relevance of midsurface in case of CAE analysis of thin-walled parts and describing motivation of choosing the topic of midsurface computation for research. It reviews traditional approaches for generating midsurfaces, comments on their advantages and limitations and brings out the critical gaps. Research objectives are laid down based on the literature review and gap analysis. Then the overview of proposed \mysystemname~is presented. 

It begins with defeaturing of the input CAD model. Irrelevant features are removed based on criteria such as sheet metal feature type, size of remnant feature portions. Apart from this, bulk negative features, called ``dormant'' features, are removed temporarily so as to simplify the CAD model without compromising on the gross shape. These dormant features are later reapplied on the generated midsurface. 

The defeatured model is further simplified by transforming the remaining sheet metal features into generalized features, representing variations of Loft feature. This generalized representation, called ``$\mathcal{ABLE}$ ({\bf A}ffine transformation, {\bf B}ooleans,  {\bf L}ofts and {\bf E}ntities)'' helps simplify the model further so that generic algorithms can be developed for generating midsurface. 

The $\mathcal{ABLE}$ model is simplified further by cellular decomposition. The cells are classified into solid and interface cells. They are delegated with the tasks of creating midsurface patches and joining them, respectively. Midsurface patches are created either by offsetting the profile face or by lofting midcurves of the profiles along the guide of the owner loft feature. Midcurve of the profile is computed by first, approximating it to polygon, then decomposing the polygon and finally, generating midcurves for each sub-polygons. Midsurface patches are joined in the interface cells by a generic logic to form a connected midsurface. The quality of output midsurface is assessed by a topological validation method which is proposed in this research. Towards the end, capabilities of \mysystemname~are demonstrated with real-life sheet metal part models. The thesis concludes by highlighting the major contributions and directions for the future work.

%In the proposed approach, first, the irrelevant features are identified and removed from the input model to compute its simplified gross shape. Remaining features, then undergo abstraction-transformation to become their corresponding generic Loft-based equivalents, each having a profile and a guide curve. The model is then decomposed into cell bodies and a graph is populated with each of the cell bodies of the nodes and fully-overlapping-surface-interfaces at the edges. The nodes are classified into midsurface-patch generating nodes (called `solid cells' or $sCell$s) and interaction-resolving nodes (interface cells' or $iCell$s). In $sCell$, a midsurface patch is generated by sweeping the midcurve of the owner-Loft-feature's profile along with its guide curve. Midsurface patches are then connected in the $iCell$s in a generic manner, thus resulting in a well-connected midsurface with minimum failures. Output midsurface is then validated topologically for correctness. At the end, practical, real-life case-studies used to demonstrate the efficacy of the approach.


\bigskip

{\bf Keywords}:  Midsurface, CAD, CAE, Cellular Decomposition, Model Simplification, Sheet Metal Features, Topological Validation, Feature Abstraction.
