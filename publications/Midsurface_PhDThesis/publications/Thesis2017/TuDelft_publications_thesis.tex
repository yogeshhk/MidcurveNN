\chapter*{List of Publications}
\addcontentsline{toc}{chapter}{List of Publications}
\setheader{List of Publications}
\label{publications}

%\vspace*{-38pt}

\begin{itemize}[noitemsep,label=\textbullet, topsep=2pt,parsep=2pt,partopsep=2pt]
\item  \textbf{International Journals}:
	\begin{enumerate} [noitemsep, topsep=2pt,parsep=2pt,partopsep=2pt]
	\item {Jan 2017}, \textit{Dimension-Reduction Technique for Polygons}, {Yogesh Kulkarni, Anil Sahasrabudhe, Mukund Kale}, {Intl. Journal of Computer Aided Engineering and Technology}, \textbf{Inderscience}.
	\item 	{Jul 2016}, \textit{Computation of Midsurface by Feature-based Simplification-Abstraction-Decomposition}, {Yogesh Kulkarni, Anil Sahasrabudhe, Mukund Kale}, {Journal of Computing and Information Science in Engineering}, \textbf{ASME}.
	\item 	{Jul 2016}, \textit{Leveraging feature generalization and decomposition to compute a well-connected midsurface}, {Yogesh Kulkarni, Anil Sahasrabudhe, Mukund Kale}, {Engineering with Computers}, \textbf{Springer}.
	\item 	{Apr 2016}, \textit{Leveraging feature information for defeaturing sheet metal feature-based CAD part model}, {Yogesh Kulkarni, Ravi Kumar Gupta, Anil Sahasrabudhe, Mukund Kale, Alain Bernard}, {CAD\&A}, \textbf{Taylor \& Francis}.
	\item 	{Apr 2015}, \textit{Topological Validation of Midsurface Computed from Sheet Metal Part}, {Yogesh Kulkarni, Anil Sahasrabudhe, Mukund Kale}, {Computer-Aided Design and Applications (CAD\&A)}, \textbf{Taylor \& Francis}.	
	\end{enumerate}
\item  \textbf{International Conferences}:
	\begin{enumerate}[noitemsep, topsep=2pt,parsep=2pt,partopsep=2pt]
	\item 	{Jun 2015}, \textit{Defeaturing Sheet Metal Part Model based on Feature Information}, {Yogesh Kulkarni, Ravi Kumar Gupta, Anil Sahasrabudhe, Mukund Kale, Alain Bernard} {Proceedings of CAD 15}, \textbf{London, Taylor \& Francis}.
	\item 	{Dec 2014}, \textit{Formulating Midsurface using Shape Transformations of Form Features},	{Yogesh Kulkarni, Anil Sahasrabudhe, Mukund Kale}, {All India Manuf. Tech. Design Research Conf. (AIMTDR)}, \textbf{IIT Guwahati}.
	\item  	{Dec 2013}, \textit{Strategies for using feature information in model simplification}, {Yogesh Kulkarni, Anil Sahasrabudhe, Mukund Kale}, {Intl. Conf. on CAE}, \textbf{IIT Madras}.	
	\item  {May 2013},	\textit{Using Features for generation of Midsurface}, {Yogesh Kulkarni, Anil Sahasrabudhe, Mukund Kale},	 {Intl. Conf. on Adv in Mech Eng}, \textbf{COEP}.
	\end{enumerate}	
\item  \textbf{National Conferences}:
	\begin{enumerate}[noitemsep, topsep=2pt,parsep=2pt,partopsep=2pt]
	\item  {Jan 2014}, \textit{Midcurves Generation Algorithm for Thin Polygons}, {Yogesh Kulkarni, Anil Sahasrabudhe, Mukund Kale}, {Natl. Conf. on Emerging Trends in Eng and Science }, \textbf{Asansol, India}.
	\end{enumerate}		
\end{itemize}
%
%%% We use the 'etaremune' environment (the reverse of 'enumerate') to get a
%%% numbered list of publications in reverse chronological order. If the list of
%%% authors is long, it might be useful to emphasize your own name with \textbf.
%\begin{etaremune}{\small
%
%\item \textbf{A.\ Einstein}, \textit{Ist die Tr\"agheit eines K\"orpers von seinem Energieinhalt abh\"angig?}, \href{http://dx.doi.org/10.1002/andp.19053231314}{Annalen der Physik \textbf{18}, 639 (1906)}.
%\item \textbf{A.\ Einstein}, \textit{Zur Elektrodynamik bewegter K\"orper}, \href{http://dx.doi.org/10.1002/andp.19053221004}{Annalen der Physik \textbf{17}, 891 (1905)}.
%\item \textbf{A.\ Einstein}, \textit{\"Uber die von der molekularkinetischen Theorie der W\"arme geforderte Bewegung von in ruhenden Fl\"ussigkeiten suspendierten Teilchen}, \href{http://dx.doi.org/10.1002/andp.19053220806}{Annalen der Physik \textbf{17}, 549 (1905)}.
%\item \textbf{A.\ Einstein}, \textit{\"Uber einen die Erzeugung und Verwandlung des Lichtes betreffenden heuristischen Gesichtspunkt}, \href{http://dx.doi.org/10.1002/andp.19053220806}{Annalen der Physik \textbf{17}, 132 (1905)}.
%
%}\end{etaremune}

