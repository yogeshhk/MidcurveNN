Computer-aided Design (CAD) models of thin-walled parts, such as sheet metal or plastics, are often dimensionally reduced to their corresponding midsurfaces for achieving quicker and reasonably accurate results in Computer-aided Engineering (CAE) analysis. However, the generation of the midsurface remains a time-consuming and predominantly manual task due to the lack of robust and automated techniques. Midsurface failures manifest as gaps, overlaps, deviations from the midway position, etc., which can require hours or even days to correct. Most existing techniques operate on the complex final shape of the model, necessitating the use of hard-coded heuristic rules developed on a case-by-case basis. The research presented here proposes to address these problems by leveraging feature parameters, made available by modern feature-based CAD applications, and by effectively utilizing them for sub-processes such as simplification, abstraction, and decomposition.

In the proposed system, features that are not part of the gross shape are initially removed from the input sheet metal feature-based CAD model. Features of the gross-shape model are then transformed into their corresponding generic feature equivalents, each having a profile and a guide curve. The abstracted model is subsequently decomposed into non-overlapping cellular bodies. The cells are classified into midsurface-patch generating cells, called 'solid cells', and patch-connecting cells, called 'interface cells'. In solid cells, midsurface patches are generated either by offsetting or by sweeping the midcurve generated from the owner-feature's profile. Interface cells join all the midsurface patches incident upon them. The output midsurface is then validated for correctness. Finally, real-life parts are used to demonstrate the efficacy of the approach.