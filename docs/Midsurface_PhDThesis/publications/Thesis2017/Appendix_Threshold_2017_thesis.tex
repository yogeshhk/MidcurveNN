Following test cases show the effect of defeaturing on the quality of the midsurface. Size threshold used here is certain percentage of the summation of face-areas of all the faces in the original CAD model.

\begin{enumerate}
\item Threshold (D) 3\% of the total part size


\begin{tabular}[h]{@{} p{0.12\linewidth}  p{0.28\linewidth} p{0.28\linewidth} p{0.28\linewidth}@{}}
\toprule
 & Model & Midsurface & Explanation \\
 \midrule
 
Original input &
\raisebox{-0.8\height}{\includegraphics[width=0.8\linewidth]{images/defeatresult_perc3_origpart}} &
\raisebox{-0.8\height}{\includegraphics[width=0.8\linewidth]{images/defeatresult_perc3_origmids}} &
Gaps in the midsurface. Two of the gaps are marked.\\

Removing sheet metal features less than threshold &
\raisebox{-0.8\height}{\includegraphics[width=0.8\linewidth]{images/defeatresult_perc3_ph1part}} &
\raisebox{-0.8\height}{\includegraphics[width=0.8\linewidth]{images/defeatresult_perc3_ph1mids}} &
Although the number of gaps have reduced, but the gaps between the surface patches is still seen. \\

Removing generic CAD features less than threshold  &
\raisebox{-0.8\height}{\includegraphics[width=0.8\linewidth]{images/defeatresult_perc3_ph2part}} &
\raisebox{-0.8\height}{\includegraphics[width=0.8\linewidth]{images/defeatresult_perc3_ph2mids}} &
Some of the gaps at the top portion are filled and the output is a bit better-connected midsurface. \\

\bottomrule
\end{tabular}

Effectiveness of Smarf with 3\% threshold by Eqn. \ref{eqn:defeaturing:effectiveness} is:

\begin{minipage}[c]{0.6\linewidth}
\begin{tabular}[h]{@{} p{0.22\linewidth} p{0.18\linewidth} p{0.21\linewidth} p{0.2\linewidth} @{}}\toprule
\textbf{Qty} & \textbf{Input} & \textbf{Phase I} & \textbf{Output}\\  \midrule
Faces  & 1434 & 610 & 327\\
Suppressed  &  & 60 & 30\\
\bottomrule
\end{tabular}
\end{minipage}
\begin{minipage}[c]{0.38\linewidth}
$pR = (1 - \frac{697}{833}) \times 100 = 16\%$
\end{minipage}


\item Threshold (D) 5\% of the total part size


\begin{tabular}[h]{@{} p{0.12\linewidth}  p{0.28\linewidth} p{0.28\linewidth} p{0.28\linewidth}@{}}
\toprule
 & Model & Midsurface & Explanation \\
 \midrule
 
Original input&
\raisebox{-0.8\height}{\includegraphics[width=0.8\linewidth]{images/defeatresult_perc5_origpart}} &
\raisebox{-0.8\height}{\includegraphics[width=0.8\linewidth]{images/defeatresult_perc5_origmids}} &
Gaps in the midsurface. Two of the gaps are marked.\\

Removing sheet metal features less than threshold &
\raisebox{-0.8\height}{\includegraphics[width=0.8\linewidth]{images/defeatresult_perc5_ph1part}} &
\raisebox{-0.8\height}{\includegraphics[width=0.8\linewidth]{images/defeatresult_perc5_ph1mids}} &
Although the number of missing gaps in the midsurface has reduced, but the gaps between the surface patches is still seen.\\

Removing generic CAD features less than threshold  &
\raisebox{-0.8\height}{\includegraphics[width=0.8\linewidth]{images/defeatresult_perc5_ph2part}} &
\raisebox{-0.8\height}{\includegraphics[width=0.8\linewidth]{images/defeatresult_perc5_ph2mids}} &
Most of the gaps are filled and the output is a better-connected midsurface. It retains all the necessary features adequately ``representing''  the gross shape. \\

\bottomrule
\end{tabular}

Effectiveness of Smarf with 5\% threshold by Eqn. \ref{eqn:defeaturing:effectiveness} is:

\begin{minipage}[c]{0.6\linewidth}
\begin{tabular}[h]{@{} p{0.22\linewidth} p{0.18\linewidth} p{0.21\linewidth} p{0.2\linewidth} @{}}\toprule
\textbf{Qty} & \textbf{Input} & \textbf{Phase I} & \textbf{Output}\\  \midrule
Faces  & 833 & 772 & 617\\
Suppressed  &  &7 & 40\\
\bottomrule
\end{tabular}
\end{minipage}
\begin{minipage}[c]{0.38\linewidth}
$pR = (1 - \frac{697}{833}) \times 100 = 26\%$
\end{minipage}

\item Threshold (D) 10\% of the total part size


\begin{tabular}[h]{@{} p{0.12\linewidth}  p{0.28\linewidth} p{0.28\linewidth} p{0.28\linewidth}@{}}
\toprule
 & Model & Midsurface & Explanation \\
 \midrule
 
Original input  &
\raisebox{-0.8\height}{\includegraphics[width=0.8\linewidth]{images/defeatresult_perc10_origpart}} &
\raisebox{-0.8\height}{\includegraphics[width=0.8\linewidth]{images/defeatresult_perc10_origmids}} &
Gaps in the midsurface. Two of the gaps are marked.\\

Removing sheet metal features less than threshold &
\raisebox{-0.8\height}{\includegraphics[width=0.8\linewidth]{images/defeatresult_perc10_ph1part}} &
\raisebox{-0.8\height}{\includegraphics[width=0.8\linewidth]{images/defeatresult_perc10_ph1mids}} &
Although the number of missing gaps in the midsurface has reduced, but the gaps between the surface patches is still seen. \\

Removing generic CAD features less than threshold  &
\raisebox{-0.8\height}{\includegraphics[width=0.8\linewidth]{images/defeatresult_perc10_ph2part}} &
\raisebox{-0.8\height}{\includegraphics[width=0.8\linewidth]{images/defeatresult_perc10_ph2mids}} &
Most of the gaps are filled. Removal of purple feature could be the domain decision. It retains all the necessary features adequately ``representing''  the gross shape. \\

\bottomrule
\end{tabular}

Effectiveness of Smarf with 10\% threshold by Eqn. \ref{eqn:defeaturing:effectiveness} is:

\begin{minipage}[c]{0.6\linewidth}
\begin{tabular}[h]{@{} p{0.22\linewidth} p{0.18\linewidth} p{0.21\linewidth} p{0.2\linewidth} @{}}\toprule
\textbf{Qty} & \textbf{Input} & \textbf{Phase I} & \textbf{Output}\\  \midrule
Faces  & 833 & 715 & 522\\
Suppressed  &  &17 & 48\\
\bottomrule
\end{tabular}
\end{minipage}
\begin{minipage}[c]{0.38\linewidth}
$pR = (1 - \frac{522}{833}) \times 100 = 37\%$
\end{minipage}
\end{enumerate}

The choice of threshold can be set to such a value where midsurface output is well-connected. In the test-cases shown above 10\% appears to be the appropriate value. Increasing the threshold to still higher value results in a well-connected midsurface but it loses the shape characteristics which need to be maintained.
