%%%%%%%%%%%%%%%%%%%%%%%%%%%%%%%%%%%%%%%%%
%Computer-Aided Design and Applications template
%mind3str@gmail.com
%%%%%%%%%%%%%%%%%%%%%%%%%%%%%%%%%%%%%%%%%

%----------------------------------------------------------------------------------------

\documentclass[9pt,a4paper]{article}
\usepackage{titling}
\usepackage[T1]{fontenc}
%\usepackage[utf8]{inputenc}
\usepackage{authblk}
\usepackage{url}
\usepackage{graphicx} % Required for the inclusion of images
\usepackage{float}
\usepackage[colorlinks,
	linkcolor=red,
	anchorcolor=blue,
	citecolor=green
]{hyperref}
\usepackage{verbatim}
\usepackage[paperwidth=7.5in, paperheight=10.3in ,top=1in, bottom=1in, left=0.6in, right=0.6in]{geometry}
\usepackage{fancyhdr}

\pagestyle{fancy}
\fancyhead[RO,RE]{\thepage}
\fancyfoot[RO,RE]{Computer-Aided Design \& Applications, 11(a), 2014, bbb-ccc \\ \copyright  2014 CAD Solutions, LLC, http://www.cadanda.com }
\renewcommand{\headrulewidth}{0pt} 
\renewcommand{\footrulewidth}{0pt}


\setlength\parindent{0pt} % Removes all indentation from paragraphs
\renewcommand{\labelenumi}{\alph{enumi}.} % Make numbering in the enumerate environment by letter rather than number (e.g. section 6)

\begin{document}

\begin{figure}
  \centering
  \includegraphics[width=0.6533\linewidth]{top.jpg}\\
\end{figure}
\title{\fontsize{12}{14.4}\textbf{Word Template for Computer-Aided Design and Applications Papers}} % Title
\author[1]{\fontsize{9}{10.8}Author A}
\author[2]{\fontsize{9}{10.8}Author D}
\author[2]{\fontsize{9}{10.8}Author E}
\affil[1]{\fontsize{9}{10.8}Department of Computer Science, University}
\affil[2]{\fontsize{9}{10.8}Department of Mechanical Engineering,  University}
\date{}
\maketitle % Insert the title, author and date
\thispagestyle{empty}



% If you wish to include an abstract, uncomment the lines below
\begin{abstract}
    \noindent This document outlines the necessary details to prepare a paper for the journal Computer-Aided Design and Applications. Authors are requested to follow all formatting instructions encoded into this MS Word file. To simplify the task of document preparation, simply enter your paper into this file by substituting equivalent titles and paragraphs of instructions with your technical text.\\
\\
    \textbf{Keywords.} select at least three words or phrases that best describe your paper.\\
   \\
    \textbf{DOI} 10.3722/cadaps.2014.xxx-yyy\\
\end{abstract}



\section{INTRODUCTION}

This template file contains all relevant information to process papers into the right format for the journal CAD and Applications. Please do not change the preset styles. Either type your paper directly into this file, substituting appropriate paragraphs and headings, or bring in your text in a plain format and then change it to the appropriate style by simply selecting the intended style from the menu in the MS-Word toolbar.
The following sections summarize the parameters used in the various pre-set styles. You should never really have to deal with these parameters directly, but only select one of the pre-formatted styles. But for completeness of documentation this information is given here, so that future managers of this template may more easily make selective changes to some of the styles.\\
This template file contains all relevant information to process papers into the right format for the journal CAD and Applications. Please do not change the preset styles. Either type your paper directly into this file, substituting appropriate paragraphs and headings, or bring in your text in a plain format and then change it to the appropriate style by simply selecting the intended style from the menu in the MS-Word toolbar.
The following sections summarize the parameters used in the various pre-set styles. You should never really have to deal with these parameters directly, but only select one of the pre-formatted styles. But for completeness of documentation this information is given here, so that future managers of this template may more easily make selective changes to some of the styles.



\subsection{MAIN BODY}
\label{definitions}
Each page is set up by this MS Word template; however, in case settings get lost, here are the details:

Page numbers have been included already and will be adjusted when the paper is desk-edited for publication. Please do not remove or edit the page numbers!
Throughout the whole paper the font Lucida Bright is used. The header block of each paper is preformatted with a header figure and two extra lines of text. This is followed by a 12-point bold title, which should be no longer than one line, or at most two, if it is absolutely necessary. Only well known abbreviations are allowed, such as NURBS, STL or IGES. The initial letters of each word should be capitalized, except for: the, a, and, or, for, in, from, etc.

This is followed by the ABSTRACT-heading and by the abstract, which both carry their own built-in spacings from the previous text elements. The abstract is indented on both sides by 0.5". It should be kept short and concise and should not exceed 10 lines.
The abstract block also carries the Keywords and the DOI entries, separated from the abstract by a single blank line. All keywords should fit into one line. Select at least three words or short phrases, all in lower case, separated by commas, as in this example:

\begin{description}
\item[Stoichiometry]
The relationship between the relative quantities of substances taking part in a reaction or forming a compound, typically a ratio of whole integers.
\item[Atomic mass]
The mass of an atom of a chemical element expressed in atomic mass units. It is approximately equivalent to the number of protons and neutrons in the atom (the mass number) or to the average number allowing for the relative abundances of different isotopes.
\end{description}


\section{SUMMARY}

The most obvious source of experimental uncertainty is the limited precision of the balance. Other potential sources of experimental uncertainty are: the reaction might not be complete; if not enough time was allowed for total oxidation, less than complete oxidation of the magnesium might have, in part, reacted with nitrogen in the air (incorrect reaction); the magnesium oxide might have absorbed water from the air, and thus weigh ``too much." Because the result obtained is close to the accepted value it is possible that some of these experimental uncertainties have fortuitously cancelled one another.

\section{ACKNOWLEDGEMENTS}

The most obvious source of experimental uncertainty is the limited precision of the balance. Other potential sources of experimental uncertainty are: the reaction might not be complete; if not enough time was allowed for total oxidation, less than complete oxidation of the magnesium might have, in part, reacted with nitrogen in the air (incorrect reaction); the magnesium oxide might have absorbed water from the air, and thus weigh ``too much." Because the result obtained is close to the accepted value it is possible that some of these experimental uncertainties have fortuitously cancelled one another.
An MS Word document can be as large as 10 MB, but still convey no more than 1 MB worth of information, if the figures are improperly processed! Figures should be inserted centered. In order to save space, you may want to insert them side-by-side if they are logically linked together. Color as well as black and white images are accepted. Each figure (group) must include a caption set in 9-point Lucida Bright. The caption is to be centered if it is short, otherwise left and right justified. Figure numbering and referencing should be done sequentially, e.g. Fig. 1., Fig. 2., etc. for single figures, and Fig. 1(a)., Fig. 1(b)., etc. for figures with multiple parts. Put a blank line before and after the figure.
An MS Word document can be as large as 10 MB, but still convey no more than 1 MB worth of information, if the figures are improperly processed! Figures should be inserted centered. In order to save space, you may want to insert them side-by-side if they are logically linked together. Color as well as black and white images are accepted. Each figure (group) must include a caption set in 9-point Lucida Bright. The caption is to be centered if it is short, otherwise left and right justified. Figure numbering and referencing should be done sequentially, e.g. Fig. 1., Fig. 2., etc. for single figures, and Fig. 1(a)., Fig. 1(b)., etc. for figures with multiple parts. Put a blank line before and after the figure.
An MS Word document can be as large as 10 MB, but still convey no more than 1 MB worth of information, if the figures are improperly processed! Figures should be inserted centered. In order to save space, you may want to insert them side-by-side if they are logically linked together. Color as well as black and white images are accepted. Each figure (group) must include a caption set in 9-point Lucida Bright. The caption is to be centered if it is short, otherwise left and right justified. Figure numbering and referencing should be done sequentially, e.g. Fig. 1., Fig. 2., etc. for single figures, and Fig. 1(a)., Fig. 1(b)., etc. for figures with multiple parts. Put a blank line before and after the figure.
An MS Word document can be as large as 10 MB, but still convey no more than 1 MB worth of information, if the figures are improperly processed! Figures should be inserted centered. In order to save space, you may want to insert them side-by-side if they are logically linked together. Color as well as black and white images are accepted. Each figure (group) must include a caption set in 9-point Lucida Bright. The caption is to be centered if it is short, otherwise left and right justified. Figure numbering and referencing should be done sequentially, e.g. Fig. 1., Fig. 2., etc. for single figures, and Fig. 1(a)., Fig. 1(b)., etc. for figures with multiple parts. Put a blank line before and after the figure.
An MS Word document can be as large as 10 MB, but still convey no more than 1 MB worth of information, if the figures are improperly processed! Figures should be inserted centered. In order to save space, you may want to insert them side-by-side if they are logically linked together. Color as well as black and white images are accepted. Each figure (group) must include a caption set in 9-point Lucida Bright. The caption is to be centered if it is short, otherwise left and right justified. Figure numbering and referencing should be done sequentially, e.g. Fig. 1., Fig. 2., etc. for single figures, and Fig. 1(a)., Fig. 1(b)., etc. for figures with multiple parts. Put a blank line before and after the figure.
An MS Word document can be as large as 10 MB, but still convey no more than 1 MB worth of information, if the figures are improperly processed! Figures should be inserted centered. In order to save space, you may want to insert them side-by-side if they are logically linked together. Color as well as black and white images are accepted. Each figure (group) must include a caption set in 9-point Lucida Bright. The caption is to be centered if it is short, otherwise left and right justified. Figure numbering and referencing should be done sequentially, e.g. Fig. 1., Fig. 2., etc. for single figures, and Fig. 1(a)., Fig. 1(b)., etc. for figures with multiple parts. Put a blank line before and after the figure.
An MS Word document can be as large as 10 MB, but still convey no more than 1 MB worth of information, if the figures are improperly processed! Figures should be inserted centered. In order to save space, you may want to insert them side-by-side if they are logically linked together. Color as well as black and white images are accepted. Each figure (group) must include a caption set in 9-point Lucida Bright. The caption is to be centered if it is short, otherwise left and right justified. Figure numbering and referencing should be done sequentially, e.g. Fig. 1., Fig. 2., etc. for single figures, and Fig. 1(a)., Fig. 1(b)., etc. for figures with multiple parts. Put a blank line before and after the figure.
An MS Word document can be as large as 10 MB, but still convey no more than 1 MB worth of information, if the figures are improperly processed! Figures should be inserted centered. In order to save space, you may want to insert them side-by-side if they are logically linked together. Color as well as black and white images are accepted. Each figure (group) must include a caption set in 9-point Lucida Bright. The caption is to be centered if it is short, otherwise left and right justified. Figure numbering and referencing should be done sequentially, e.g. Fig. 1., Fig. 2., etc. for single figures, and Fig. 1(a)., Fig. 1(b)., etc. for figures with multiple parts. Put a blank line before and after the figure.
An MS Word document can be as large as 10 MB, but still convey no more than 1 MB worth of information, if the figures are improperly processed! Figures should be inserted centered. In order to save space, you may want to insert them side-by-side if they are logically linked together. Color as well as black and white images are accepted. Each figure (group) must include a caption set in 9-point Lucida Bright. The caption is to be centered if it is short, otherwise left and right justified. Figure numbering and referencing should be done sequentially, e.g. Fig. 1., Fig. 2., etc. for single figures, and Fig. 1(a)., Fig. 1(b)., etc. for figures with multiple parts. Put a blank line before and after the figure.
An MS Word document can be as large as 10 MB, but still convey no more than 1 MB worth of information, if the figures are improperly processed! Figures should be inserted centered. In order to save space, you may want to insert them side-by-side if they are logically linked together. Color as well as black and white images are accepted. Each figure (group) must include a caption set in 9-point Lucida Bright. The caption is to be centered if it is short, otherwise left and right justified. Figure numbering and referencing should be done sequentially, e.g. Fig. 1., Fig. 2., etc. for single figures, and Fig. 1(a)., Fig. 1(b)., etc. for figures with multiple parts. Put a blank line before and after the figure.
An MS Word document can be as large as 10 MB, but still convey no more than 1 MB worth of information, if the figures are improperly processed! Figures should be inserted centered. In order to save space, you may want to insert them side-by-side if they are logically linked together. Color as well as black and white images are accepted. Each figure (group) must include a caption set in 9-point Lucida Bright. The caption is to be centered if it is short, otherwise left and right justified. Figure numbering and referencing should be done sequentially, e.g. Fig. 1., Fig. 2., etc. for single figures, and Fig. 1(a)., Fig. 1(b)., etc. for figures with multiple parts. Put a blank line before and after the figure.

\section{REFERENCES}

\bibliographystyle{model1-num-names}
%\bibliography{ref.bib}
\bibliography{../Common/Bibliography}
%----------------------------------------------------------------------------------------


\end{document}