%\begin{abstract}


Computer-aided Design (CAD) models of thin-walled solids such as sheet metal or plastic parts are often reduced dimensionally to their corresponding midsurfaces for quicker and fairly accurate results of Computer-aided Engineering (CAE) analysis. Computation of the midsurface is still a time-consuming and mostly, a manual task due to lack of robust-automated techniques. Most of the existing techniques work on the final shape (typically in the form of Boundary representation, B-rep). Complex B-reps make it hard to detect sub-shapes for which the midsurface patches are computed and joined, forcing usage of hard-coded geometric/heuristic rules, developed on a case-to-case basis. Midsurface failures manifest in the form of gaps, overlaps, not-mimicking-input-model, etc., which take hours or even days to correct.  The research presented here proposes to address these problems by leveraging feature-information, available in the modern CAD models,         and by effectively using techniques like simplification, abstraction and decomposition, in a generic manner. 

In the proposed approach, first, the irrelevant features are identified and removed from the input model to compute its simplified gross shape. Remaining features, then undergo abstraction-transformation to become their corresponding generic Loft-based equivalents, each having a profile and a guide curve. The model is then decomposed into cellular bodies and a graph is populated with each of the cellular bodies at the nodes and fully-overlapping-surface-interfaces at the edges. The nodes are classified into midsurface-patch generating nodes (called `solid cells' or $sCell$s) and interaction-resolving nodes (`interface cells' or $iCell$s). In $sCell$, a midsurface patch is generated by sweeping the midcurve of the owner-Loft-feature's profile along with its guide curve. Midsurface patches are then connected in the $iCell$s in a generic manner, thus resulting in a well-connected midsurface with minimum failures. Output midsurface is then validated topologically for correctness. At the end, a practical, real-life part is used to demonstrate the efficacy of the approach.

\vspace{0.25cm}

\textbf{keyword}: Midsurface, CAD, CAE, Cellular Decomposition, Model Simplification, Sheet Metal Features, Topological Validation, Feature Abstraction

%{\bf Keywords}: CAD, CAE, FEA, Model Simplification, Midsurface, Feature-based Design
%\end{abstract}