Complex models prepared in CAD applications are often simplified before using them in downstream applications like CAE, shape matching, multi-resolution modeling, etc. In CAE, the thin-walled models are often abstracted to a midsurface for quicker analysis. Computation of the midsurface has been observed to be effective when the original model is defeatured to its gross shape. 

Defeaturing in this paper proposes a novel approach for computation such gross shape and it works in two phases. First, a proposed sheet metal feature-based classification scheme (taxonomy) is used to determine the suppressibility of the features. Second, a method based on the size of remnant portions of the feature volume is developed to determine the eligibility for suppression. Case studies are presented to demonstrate the efficacy of the proposed approach. It shows that even after substantial reduction in the number of faces the gross shape retains all the important features needed for computation of a well-connected midsurface.