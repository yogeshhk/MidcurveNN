During the initial stages of iterative design process, a quick CAE (Computer-aided Engineering) analysis of the CAD (Computer-aided Design) models is needed. To reduce the computational resources and time needed for such analysis, the models are often simplified by removing the irrelevant details and are abstracted by reducing the dimension, wherever appropriate. Thin-walled parts, such as sheet metal parts are often abstracted to a set of surfaces lying midway, called midsurface. The midsurface is expected to mimic the shape of the original solid, both geometrically and topologically. Widely-used methods of accessing the quality of the midsurface are geometric. Hausdorff distance from the midsurface its original solid is computed to find the gaps and medial-ness. Accuracy of such methods depends on the sampling as well as on the complexity of the surface representation, making them computationally intensive and error-prone.

This paper provides a topological method for verification, which is computationally simple and robust. A novel topological transformation relationship has been derived between a sheet metal part (solid) to its midsurface (surface), in both directions (solid-to-surface and surface-to-solid) which can be used to compare the predicted vs actual entities. Simple as well as practical shapes have been tested to prove the efficacy of the newly-derived formulation.
