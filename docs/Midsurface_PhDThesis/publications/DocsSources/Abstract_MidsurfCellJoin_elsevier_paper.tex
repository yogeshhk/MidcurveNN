Computer-aided Design (CAD) models of thin-walled solids such as sheet metal/plastic parts, are often abstracted to their midsurfaces for Computer-aided Engineering (CAE) analysis. The reason being, 2D surface elements placed on the midsurface provides fairly accurate results, while requiring far lesser computational resources-time compared to the analysis using 3D solid-elements. Existing automatic methods of midsurface-computation are not reliable and robust. They result in an ill-connected midsurface having missing patches, gaps, overlaps etc.  Errors need to be corrected mostly by a manual, time-consuming process requiring hours to even days. Thus, an automatic and robust technique for computation of a well-connected midsurface is a need of the hour. %As most of the existing approaches work on the final Boundary-representation (B-rep) of the model, it becomes challenging to divide the complex model into simpler sub-volumes and resolve interactions amongst the midsurface patches generated by each sub-volume.

This paper proposes a method which, instead of working on the complex final solid shape (B-rep, Boundary Representation), leverages feature information in modern CAD models and techniques such as simplification, abstraction and cellular decomposition. Here, first, the model is simplified by removing the irrelevant features, to generate a ``gross shape''.  Features are then abstracted to their corresponding generic loft-feature equivalents. Model is decomposed to form cellular bodies having respective owner-loft-features. A graph is populated with the cellular bodies at the graph-nodes and are  classified into midsurface-patch generating nodes (called `solid cells' or $sCell$s) and interaction-resolving nodes (called `interface cells' or $iCell$s). Using owner-loft-feature's parameters, $sCell$s compute their own midsurface patches and using a generic logic, they get connected appropriately in the $iCell$s, resulting in a well-connected midsurface.

Towards the end of the paper, we demonstrate the efficacy of this approach by computing well-connected midsurfaces of various academic models as well as of a real-life sheet metal part.