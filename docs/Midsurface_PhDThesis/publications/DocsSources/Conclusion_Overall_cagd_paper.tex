\section{Conclusion}

%Simplification by removing the irrelevant data, decomposition to reduce complexity of the domain, and abstraction/generalization to devise a generic logic, have been core themes of this work to compute a well-connected midsurface.

Computation of midsurface is one of the most popular simplification techniques for CAE analysis of  thin-walled models. Study of the existing techniques revealed critical insights into the problems of the  midsurface results (gaps, overlaps, lack of isomorphism, etc.), such as failures in the face-pair detection and interactions. Interactions were studied to conclude that the cellular decomposition can be used to develop a uniform logic for joining the midsurface patches.

In the proposed approach, a feature-based  CAD model of thin-walled solid is simplified first by defeaturing techniques to result into ``gross shape'' which is found to reduce failures in the  downstream process. The remaining features are then converted to their equivalent generic form of Sweep/Loft.  Model is decomposed so that computing midsurface patches from each sub-volume, as well as connecting them becomes far easier.  Thus, the proposed methodology scores over existing approaches not only in terms of simplifying the complexities to a great extent, but also being applicable over a wide variety of geometries and connection types. % (Table  \ref{tab:defeat_test}). 

%\begin{table}[!htb]
%  \centering 
%\resizebox{0.8\linewidth}{!}{ 
%\begin{tabular}[htp]{@{} p{0.15\linewidth} p{0.22\linewidth}  p{0.22\linewidth} p{0.22\linewidth} p{0.22\linewidth}@{}}
%\toprule
%\textbf{Researcher}  &	\textbf{Method} &	 \textbf{Shortcomings}  &	 \textbf{Our Approach}\\ \midrule
%\textbf{Chong et al.} \cite{Chong2004}  & 
%Uses concave edge decomposition. Midcurves by collapsing edge pairs. If they form a loop, creates a midsurface patch. & 
%Hard-coded inequalities/values to detect edge-pairs. Connection logic is not generic and comprehensive. &
%A generic treatment for the computation of midcurves, midsurface patches and their connections.
%\\ \midrule
% \textbf{Boussuge et al.} \cite{Boussuge2014}  & 
% Generative decomposition. Recognizes Extrudes of each sub-volume. Creates midsurface patches in each sub-volume and connects them together. &
% 
% \begin{itemize}[noitemsep,nosep,leftmargin=*]
%\item No fillets/chamfers.
%\item Only Additive cells.
%\item Only Extrudes with Analytical surfaces.
%\item Expensive MAT to detect thin profiles.
%\item Works only on Parallel and Orthogonal connections.
%\end{itemize}
%&
%\begin{itemize}[noitemsep,nosep,leftmargin=*]
%\item No such restriction.
%\item Re-inserts -ve cells.
%\item Generic Sweep extend-able to Loft.
%\item Simple rules of size of profile/guide.
%\item Generic logic for any numbers/types of connections.
%\end{itemize}
%
%\\ \midrule
%
%\end{tabular}
%}
% \caption{Comparison with Other Methods}
%  \label{tab:defeat_test}
%\end{table}


Although individual functionalities such as defeaturing, abstraction and decomposition processes may show certain shortcomings, but overall, the proposed approach seems to work well and is able to compute a well-connected midsurface in a robust, definitive and generic manner, with minimal failures.  

The superiority of the proposed approach is in the manner in which it has abstracted both the sub-problems, i.e., computation of midsurface patches and resolving interactions amongst them. No heuristic rules based on specific surface-types or connection configurations are used, making the whole process generic and adaptable in a wide variety of configurations.


