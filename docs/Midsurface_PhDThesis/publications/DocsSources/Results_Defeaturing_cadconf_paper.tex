
%----------------------------------------------------------------------------------------------------------------
\section{Results} \label{results}

Following test case shows effect of defeaturing on the quality of the midsurface. Size threshold used here is certain percentage of the summation of face-areas of all the faces in the original body.

\begin{enumerate}
\item Threshold (D) 3\% of the total part size


\begin{tabular}[h]{@{} p{0.12\linewidth}  p{0.28\linewidth} p{0.28\linewidth} p{0.28\linewidth}@{}}
\toprule
 & Model & Midsurface & Explanation \\
 \midrule
 
Original/Input &
\raisebox{-0.8\height}{\includegraphics[width=0.8\linewidth]{..//Common/images/defeatresult_perc3_origpart}} &
\raisebox{-0.8\height}{\includegraphics[width=0.8\linewidth]{..//Common/images/defeatresult_perc3_origmids}} &
Gaps in the midsurface. Two of the gaps are marked (blue and red).\\

Output of Phase I and input to Phase II &
\raisebox{-0.8\height}{\includegraphics[width=0.8\linewidth]{..//Common/images/defeatresult_perc3_ph1part}} &
\raisebox{-0.8\height}{\includegraphics[width=0.8\linewidth]{..//Common/images/defeatresult_perc3_ph1mids}} &
Although the number of missing gaps in the midsurface has reduced (red gap is filled), but the gaps between the surface patches (blue gap) is still seen. These gaps are marked.\\

Output of Phase II &
\raisebox{-0.8\height}{\includegraphics[width=0.8\linewidth]{..//Common/images/defeatresult_perc3_ph2part}} &
\raisebox{-0.8\height}{\includegraphics[width=0.8\linewidth]{..//Common/images/defeatresult_perc3_ph2mids}} &
Some of the gaps (blue gap top portion) are filled and the output is a bit better-connected midsurface. It retains many of the necessary features adequately ‘representing’ the gross shape. \\

\bottomrule
\end{tabular}

Effectiveness of Smarf with 3\% threshold, based on the criterion defined by Eqn. \ref{val} is:

\begin{minipage}[c]{0.6\linewidth}
\begin{tabular}[h]{@{} p{0.22\linewidth} p{0.18\linewidth} p{0.21\linewidth} p{0.2\linewidth} @{}}\toprule
\textbf{Qty} & \textbf{Input} & \textbf{Phase I} & \textbf{Output}\\  \midrule
Faces  & 1434 & 610 & 327\\
Suppressed  &  & 60 & 30\\
\bottomrule
\end{tabular}
\end{minipage}
\begin{minipage}[c]{0.38\linewidth}
$pR = (1 - \frac{697}{833}) \times 100 = 16\%$
\end{minipage}


\item Threshold (D) 5\% of the total part size


\begin{tabular}[h]{@{} p{0.12\linewidth}  p{0.28\linewidth} p{0.28\linewidth} p{0.28\linewidth}@{}}
\toprule
 & Model & Midsurface & Explanation \\
 \midrule
 
Original/Input &
\raisebox{-0.8\height}{\includegraphics[width=0.8\linewidth]{..//Common/images/defeatresult_perc5_origpart}} &
\raisebox{-0.8\height}{\includegraphics[width=0.8\linewidth]{..//Common/images/defeatresult_perc5_origmids}} &
Gaps in the midsurface. Two of the gaps are marked (blue and red).\\

Output of Phase I and input to Phase II &
\raisebox{-0.8\height}{\includegraphics[width=0.8\linewidth]{..//Common/images/defeatresult_perc5_ph1part}} &
\raisebox{-0.8\height}{\includegraphics[width=0.8\linewidth]{..//Common/images/defeatresult_perc5_ph1mids}} &
Although the number of missing gaps in the midsurface has reduced (red gap is filled), but the gaps between the surface patches (blue gap) is still seen. These gaps are marked.\\

Output of Phase II &
\raisebox{-0.8\height}{\includegraphics[width=0.8\linewidth]{..//Common/images/defeatresult_perc5_ph2part}} &
\raisebox{-0.8\height}{\includegraphics[width=0.8\linewidth]{..//Common/images/defeatresult_perc5_ph2mids}} &
Most of the gaps (blue gap full) are filled and the output is a better-connected midsurface. It retains all the necessary features adequately ‘representing’ the gross shape. \\

\bottomrule
\end{tabular}

Effectiveness of Smarf with 5\% threshold, based on the criterion defined by Eqn. \ref{val} is:

\begin{minipage}[c]{0.6\linewidth}
\begin{tabular}[h]{@{} p{0.22\linewidth} p{0.18\linewidth} p{0.21\linewidth} p{0.2\linewidth} @{}}\toprule
\textbf{Qty} & \textbf{Input} & \textbf{Phase I} & \textbf{Output}\\  \midrule
Faces  & 833 & 772 & 617\\
Suppressed  &  &7 & 40\\
\bottomrule
\end{tabular}
\end{minipage}
\begin{minipage}[c]{0.38\linewidth}
$pR = (1 - \frac{697}{833}) \times 100 = 26\%$
\end{minipage}

\item Threshold (D) 10\% of the total part size


\begin{tabular}[h]{@{} p{0.12\linewidth}  p{0.28\linewidth} p{0.28\linewidth} p{0.28\linewidth}@{}}
\toprule
 & Model & Midsurface & Explanation \\
 \midrule
 
Original/Input &
\raisebox{-0.8\height}{\includegraphics[width=0.8\linewidth]{..//Common/images/defeatresult_perc10_origpart}} &
\raisebox{-0.8\height}{\includegraphics[width=0.8\linewidth]{..//Common/images/defeatresult_perc10_origmids}} &
Gaps in the midsurface. Two of the gaps are marked (blue and red).\\

Output of Phase I and input to Phase II &
\raisebox{-0.8\height}{\includegraphics[width=0.8\linewidth]{..//Common/images/defeatresult_perc10_ph1part}} &
\raisebox{-0.8\height}{\includegraphics[width=0.8\linewidth]{..//Common/images/defeatresult_perc10_ph1mids}} &
Although the number of missing gaps in the midsurface has reduced (red gap is filled), but the gaps between the surface patches (blue gap) is still seen. These gaps are marked..\\

Output of Phase II &
\raisebox{-0.8\height}{\includegraphics[width=0.8\linewidth]{..//Common/images/defeatresult_perc10_ph2part}} &
\raisebox{-0.8\height}{\includegraphics[width=0.8\linewidth]{..//Common/images/defeatresult_perc10_ph2mids}} &
Most of the gaps (blue gap full) are filled. Removal of purple feature could be the domain decision. It retains all the necessary features adequately ‘representing’ the gross shape. \\

\bottomrule
\end{tabular}

Effectiveness of Smarf with 10\% threshold, based on the criterion defined by Eqn. \ref{val} is:

\begin{minipage}[c]{0.6\linewidth}
\begin{tabular}[h]{@{} p{0.22\linewidth} p{0.18\linewidth} p{0.21\linewidth} p{0.2\linewidth} @{}}\toprule
\textbf{Qty} & \textbf{Input} & \textbf{Phase I} & \textbf{Output}\\  \midrule
Faces  & 833 & 715 & 522\\
Suppressed  &  &17 & 48\\
\bottomrule
\end{tabular}
\end{minipage}
\begin{minipage}[c]{0.38\linewidth}
$pR = (1 - \frac{522}{833}) \times 100 = 37\%$
\end{minipage}
\end{enumerate}
