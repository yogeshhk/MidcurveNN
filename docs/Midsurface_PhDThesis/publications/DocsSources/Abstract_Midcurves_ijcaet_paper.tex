\begin{abstract}

Applications such as pattern recognition, shape matching, finite element analysis need lower dimensional representation of a higher dimensional shape.  Skeleton provides such one dimensional representation for a 2D or 3D shape. It can be computed for various types of inputs, such as images, solids, sketches. Although there are  various methods available for computation of the skeleton, their appropriateness depends on the requirement posed by their applications. Some focus on exact computation, some are approximate, some aim at faithful backward-reconstruction; whereas some focus at proper representation for human perception.

This paper focuses on a method for creation of a particular type of skeleton, called {\em Midcurve} (a skeleton that lies at the midway of a shape) for planar polygon. This method is based on the Divide-and-Conquer strategy and works in two phases - decomposition and midcurve creation. Towards the end of the paper some test-case results are presented along with the conclusions.

\end{abstract}
