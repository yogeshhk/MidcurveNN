\section{Midsurface using features}

In the proposed approach, (as listed in Algorithm \ref{alg2}) Midsurface is generated at individual feature level. At each feature step, as shapes are relatively simpler than the final shape, creation of Midsurface becomes relatively straight-forward. For a typical work-flow, here are some of the steps/rules that are applied:

\begin{enumerate}
\item Sketches of the profile based features are extracted first. The mid-curves are then created in the sketch profile itself. If information regarding Sketch sub-commands like 'Sketch-Offset', 'Circle', 'Rectangle', are available, they can be leveraged to create standard mid-curves. If such information is not available, slender portions need to be identified using MAT and then post-processed MAT curves can be used as mid-curves.
\item In operations like 'Extrude', 'Revolve', 'Sweep', a profile is swept to create a tool body which then gets boolean-ed with the base part. The mid-curves get swept similarly during feature creation. Along with the tool body thus formed Midsurface-patches are boolean-ed as well. Midsurface gets built as part is getting built. 
\item One can apply de-featuring rules at each stage as well, as demonstrated in the last step of Table ~\ref{DryRun}. Here small Holes are ignored in the Midsurface output. 

\item {\bf Proposed Algorithm}

	\begin{enumerate}
	
	\item Traverse the Feature tree one by one
	\item For each feature, decide its contribution to the Midsurface model
	
	\begin{enumerate}
	\item {\bf 2D Profiles }
		\begin{enumerate}
		\item Use internal Sketch commands information, like Offset-Curve, Mirror, Fillet to arrive at the Mid-curves profile
		\item If such command's information is not available then try other approaches like MAT, Parametric Equation, and Chordal Axis Transform etc.
		\end{enumerate}
	\item {\bf Primitives}: For positive primitive shapes (protrusions or stand-alones) create a pre-defined Midsurface
	\item {\bf Sweep based Features (Extrude Sweep Revolve Loft)}: For positive primitive shapes (protrusions or stand-alones) sweep Mid-Curves profile similar that of the parent feature.
	\item {\bf Boolean (External)}: Extend and Trim the Midsurfaces coming from Base part and Tool bodies.
	\item {\bf Tweak}: Features like Fillet, Chamfer, ignore if small as part of the de-featuring process.
	\item {\bf Shell}: Use half the Shell distance and offset the same faces as in the parent feature to arrive at the Midsurface.
	\item Intersect Midsurface with the feature faces to make sure that it lies inside the feature volume.
	\item Assign the Thickness values on the Midsurface with pre-defined grid spacing.

	\end{enumerate}
	\item Check for validity of the Midsurface output, especially for gaps and overlapping surfaces and correct the problems.

	\end{enumerate}
\end{enumerate}

\begin{algorithm}

	\caption{Midsurface creation on per feature basis }

	\label{alg2}

	\begin{algorithmic}

		\REQUIRE A CAD model (preferably de-featured) with access to the feature tree

		\WHILE{End of tree has  not reached}

			\STATE Get the current feature.

			\IF {feature type is Sketch}
				\STATE  If information regarding Sketch sub-commands like 'Sketch-Offset', 'Circle', 'Rectangle', are available, they can be leveraged to create standard mid-curves. 
				\STATE If such information is not available, slender portions need to be identified using MAT and then post-processed MAT curves can be used as mid-curves.
			\ENDIF

			\IF {feature type is Primitives}
				\STATE  For positive primitive shapes (protrusions or stand-alones) create a predefined Midsurface
			\ENDIF

			\IF {feature type is  Sweep based Features (Extrude Sweep Revolve Loft)}
				\STATE For positive primitive shapes (protrusions or stand-alones) sweep Mid-Curves profile similar that of the parent feature.
			\ENDIF

			\IF {feature type is Boolean (External)}
				\STATE Extend and Trim the Midsurface coming from Base part and Tool bodies.
			\ENDIF

			\IF {feature type is Tweak}
				\STATE Features like Fillet, Chamfer, ignore if small as part of the de-featuring process.
			\ENDIF

			\IF {feature type is Shell}
				\STATE  Use half the Shell distance and offset the same faces as in the parent feature to arrive at the Midsurface.
			\ENDIF

			\STATE Intersect Midsurface with the feature faces to make sure that it lies inside the feature volume.

			\STATE Assign the Thickness values on the Midsurface with predefined grid spacing.

		\ENDWHILE
		\STATE Midsurface gets built as part gets developed feature by feature.
		\STATE Check for validity of the Midsurface output, especially for gaps and overlapping surfaces and correct the problems

	\end{algorithmic}

\end{algorithm}
