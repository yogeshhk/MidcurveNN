\chapter{Editing}


	Editing is one of the most complex tasks in a software system largely
	because it involves ,in most cases , some sort of human interference.Proper
	care needs to be taken in the editing process to perform the modifications
	asked by the user. Editing process can be broken down in following major
	tasks.

	\begin{enumerate}

	\item
	Selecting an object to modify
	\item
	Accepting the parameters to modify the selected entity 
	\item
	Validating the specified  parameters and also its applicability to the
		problem domain.
	\item
	Actually performing the changes as asked by the user.
	\item
	Displaying them correctly.

	\end{enumerate}

	A subtask, of above mentioned steps could be an UNDO operation which
	brings user back to the state of previous action's result.


	\section{ Feature Editing }

	There are two types of entities involved in this implementation.
	Geometry and Topology. Geometrical editing involves changing the shape
	parameters of the blocks and Topological editing involves change in the
	topological relations between the blocks.

	Graph structure used to model the component proves especially advantageous
	in editing algorithms as almost all the core processes boil down to change
	in graph topology and internal node data.~\cite{Howard}

	\subsection{Geometrical Editing} 

		The shape of the block can be changed in various fashions and in almost
	all cases is dependent on the data structure used to model it. Feature
	based design models the block in parametric representatioins and thus shape
	parameters become obvious choice for editing.Geometric editing in feature
	based design involves much more work than just changing the dimensions of
	the block selected by the user.The complexity comes as blocks are linked
	to each other by the topological links. As it is must to maintain the
	topological relations the changes in the dimensions of one block may
	affect changes in the dimension of all blocks that are connected to it
	and also the blocks subsequently connected to them.

		Geometrical editing is performed at two levels.

		\begin{enumerate}

		\item
		External, which involves changes in external block shapes
		\item
		Internal, which involves changes in shape parameters specific to
			the shape involved.

		\end{enumerate}

		We will take a close look at how External editing is performed.At a
	time only one external block ( called just block hereafetr unless specified)	parameter is allowed to be changed. Multiple changes can be performed one 
	by one.Each Block axis is characterized by 4 parameters. MIN, MID, MAX and 
	Length.
	For changing one of them needs specification of other parameter that needs
	to be fixed so that the remaining two can be calculated internally.

	EXAMPLE : of maintanence of CFM with explanation.












	(see 25 Iyad/41 Venky)

	The example explained was a simple one as some of the components are
	linked such that the n th block is by some path connected to 1 st block
	forming what is called as Cycle in the Graph terminology .Editing involves 
	different set of algorithms as changes from the original modified block
	come to the same one after propagating through sets of block and may
	result in inconsistency in geometry.	

	EXAMPLE : of cycle in component and inconsistency that may cause








	(prepare your own, 28 Iyad)

	\section{Cycle detection in Graphs}

    \subsection{Definitions}

	\begin{itemize}

	\item
    A undirected graph G is "connected" if there is at least one
    path between every pair of vertices vi and vj in G.
	\item
    A directed graph G is connected if the undirected graph obtained by
    ignoring the edge directions in G is connected.
	\item
    A directed graph is said to be strongly connected if for every pair of
    vertices vi and vj there exits at least one directed path form vj to vi.

	\end{itemize}

	The simplest problem, of course, is to determine whether a given graph,
	directed or undirected contains a cycle. Depth first search suffices
	to answer the question for both directed and undirected graphs : a graph
	or a digraph has a cycle if and only if the DFS-tree contains a back edge.	
	A slightly harder problem is to determine a minimal set of cycles of an
	undirected graph from which all of the cycles in the graph can be 
	constructed ( a so called fundamental set of cycles).It is still harder
	to determine all the cycles of a digraph.[BOOK/graph/Reingold] With the
	combination of the fundamental cycles all other cycles can be derived.


		At first a spanning tree of the graph is generated with Depth first
	search. When algorithm finds a back edge ( an edge is called Back edge
	because they lead back to vertices that have been previously visited),
	the cycle consists of the tree edges from the current node to the
	destination node of the edge and the current edge. So as to collect proper
	edges of tree a stack is used while performing depth first search which
	stores all the vertices on the path from the root to the vertex currently
	being visited. When the back edge is encountered, the cycle formed
	consists of that back edge and the edges connecting the vertices on the
	top of the stack.


		Any algorithm for generating all the cycles of a
	digraph can also be used for undirected graphs by converting the undirected
	graph into a digraph by replacing each edge with two oppositely directed
	edges between the same pair of vertices.~\cite{Rein}


		The graph structure used here is actually bidirectional with directed
	edges in both directions. Precaution is taken to remove all the cycles with
	two edges and  thereby avoiding cycles between two adjacent nodes.


		Extra care needs to be taken so that every cycle detected is new one and
	not just combination of the already found ones.The process employed in
	detection of cycles is based on CYCLE algorithm ~\cite{Rein} and 
	is explained briefly here.


	All the nodes are numbered in an ascending order. This is done at the
	creation stage only. We begin the search at a vertex s and build a(directed)
	path (s,v1,v2,v3....,vk), such that vi > s, 1<= i <=k. A cycle is reported
	only when the next vertex vk+1 = s. After generating the cycle(s ,v1,v2,...
	vk,s), we explore the next edge going out of vk. If all the edges going out
	of vk have been explored, we back up to the previous vertex vk-1 and try
	extending from it and so on. This process continues until we finally try
	back up past the starting vertex s;all the cycles containing vertex s have
	been generated. This process is repeated for all the nodes if we want to
	find out all the cycles. In the current implementation the cycles having
	the edited node are only presented to user for selection.

	All the vertices are made "unavailable"( by setting available flag in the
	node as FALSE as soon as v is appended on the current path) 
	while searching with graph
	rooted at s so as to avoid the traversing of the cycles origina-
	ting at vi.It remains "unavailable" until search backs up past v to the
	previous vertex on the current path. A record is kept of the predecessors
	of all unavailable vertices that are not on the current path by maintaining
	a separate set for each vertex. When u becomes available so do all its
	unavailable predecessors that are not in the current path. Thus all the
	elements in that set are made available. As these members are made 
	available again, their unavailable predecessors that are not the current 
	path are made available again and so on.This recursive procedure is called
	UNMARK in the implementation.


    Every cycle in a digraph lies entirely within a strongly connected
    component. So at start graph is decomposed into strongly connected
    components.


	After finding strong components, the above mentioned procedure is applied
	to every component at a time. In the current implementation , cycle
	detection is done only for the component that contains the edited node.


	INSERT : Flow chart for Editing(Venky)


	\subsection{Complete Procedure}

	\begin{itemize}

	\item
   Set current component
	\item
   Select Block ( Node ) for editing
    i.e length to be edited and the constraint plane

	\item
   FINDCYCLES for all the nodes ( actually for the time being I am
    doing it for the edited node only to save the extra computations

	\item
   Processthe cycles to remove all the cycles having path of 2 nodes
    only and removing the duplicate Cycles

	\item
   Present the cycles to the user to pick for processing

	\item
   If there are NO cycles then he indicates such thing or else if he
    he wants cycle processing then he has to give the modification node

	\item
   while traversing the cycle from both sides go on putting the  nodes
    in the queue (respective to the axis chosen)

	\item
   If you don-t want cycles then you come to following steps directly

	\item
   Put this node in the Queue ( of the type of the length to be edited)

	\item
	
	\begin{verbatim}

    for( XQ, YQ, ZQ){

        while( not end of Q){

            Set current node as Base Node

            Modifylength();

            // Searches for "lengthid" in the Length-Listid and
            // finds other links.
            // if edited length is different and NOT fixed Add
            // those other nodes to the Qs of that Length type

            if all the adjacent nodes of Base Node are Resolved

                mark the base node as VISITED
                Remove this Base node from the Q

            Go to the Next Node in the Q

        }

    }// end of all Qs

	\end{verbatim}

	\item
   Done.

	\end{itemize}


