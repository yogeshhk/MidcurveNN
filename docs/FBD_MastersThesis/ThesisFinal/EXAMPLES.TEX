\chapter{Examples}

	\section{Introduction}

		In the previous chapter, a detailed description of the user
		interface for a prototype FBD system for orthohedral components
		was presented.
		This chapter demonstrates some of the capabilities developed in
		the current prototype system with the help of few examples. 
		All the examples indirectly demonstrate the ``Creation'' capability
		also.

	\section{Dimensioning}

		The example shown in Fig.~\ref{ex1dim1} depicts how a component gets 
		dimensioned by default. Internally,
		the dimensioning scheme uses the graph data structures for reference
		dimension graph(RDG) and specified dimension graph(SDG) which are also 
		shown in the Fig.~\ref{ex1dim1}.

        \begin{figure}[htbp]
          %  \centerline{\psfig{figure=ex1dim1.ps,width=4.0in,height=5.0in}}
            \caption{Dimensioning the component}
            \label{ex1dim1}
        \end{figure}

		The Fig.~\ref{ex1dim1} shows default dimensioning scheme which is a
		datum dimensioning scheme along X axis.

		Operations performed by the user to change the dimensioning scheme
		are reflected only in the specified dimension graph while the reference
		dimension graph remains unchanged. 
		The operations are listed below.

		\begin{enumerate}
		\item
					{\em Specifying the dimension between two planes} :

                    If the user wants to specify the dimension ``e'' between 
                    (say) plane 6 and plane 7, following actions are performed :

        \begin{figure}[htbp]
         %  \centerline{\psfig{figure=ex1dim2.ps,width=4.0in,height=5.0in}}
            \caption{Specifying the dimension between two planes}
            \label{ex1dim2}
        \end{figure}


                    When the input is given, the two selected primitives are
                    located in the RDG and are checked for connectedness to
                    ensure that a path exists between them. If no such path is
                    found, an error message is issued since the dimension
                    cannot be specified. If a path is found, the value of the
                    dimension is computed by adding (or subtracting)
                    and is checked with the value specified by the
                    user. If this matches, checks are made to see if the
                    modification will lead redundancy in the 
                    dimensioning. If this check is passed,
                    the SDG is modified while the RDG is kept unchanged.
					The modified SDG is shown in Fig.~\ref{ex1dim2}, where
					a new arc is added between nodes 6 and 7 and the arc
					between nodes 1 and 7 is deleted as it would have created
					redundancy represented by a cycle.

		\item

    				{\em Change in Datum} :

    				The user may want to change the datum plane to plane 7
    				in the mentioned dimensioning scheme so that all the other
					planes are dimensioned with plane 7 as datum plane. 
					This is accomplished 
					by removing all arcs emanating from previous datum plane 1
					and making them emanate from the new datum plane 7 in
					the specified dimension graph.
					The modified SDG is shown in Fig~\ref{ex1dim3}.

        \begin{figure}[htbp]
         %   \centerline{\psfig{figure=ex1dim3.ps,width=4.0in,height=5.0in}}
            \caption{Change in Datum}
            \label{ex1dim3}
        \end{figure}

		\item

    				{\em Query of Dimension} :

    				If the user wants to query the distance between (say) 
					plane 2 and
					plane 7. A graph traversal algorithm first locates the 
					two primitives in the SDG, finds the unique path between 
					them and computes the distance by summing the dimension 
					attributes on the arcs that make up the path. 
					The computed distance is displayed to the user.

		\end{enumerate}

	\section{Bspline Dimensioning}

	The following example of a vice demonstrates the capability of 
	dimensioning the Bspline extruded block. The vice consists of two 
	rectangular blocks and a Bspline extruded block.
    Formation of the RDG and the SDG requires contribution of the dimensioning
    planes from individual blocks. These planes can be used to specify a
    dimension with any other dimensioning plane. It is necessary that the
    B-spline extruded block provides not only the dimensioning planes 
	corresponding to the
    external rectangular block but also the dimensioning planes corresponding
    to selected points on the profile of the Bspline.
	The example shows dimensions along X axis (see Fig.~\ref{ex2dim}).


        \begin{figure}[htbp]
          %  \centerline{\psfig{figure=ex2dim.ps,width=4.0in,height=5.0in}}
            \caption{Bspline Dimensioning}
            \label{ex2dim}
        \end{figure}


        The dimensioning procedure is as follows :
            \begin{itemize}

            \item
            For pre-specified intervals (say $n$), values $x_{1}$ to $x_{n}$
            are generated at equal intervals, where $x_{1}$ corresponds to the
            Xmin plane and $x_{n}$ corresponds to the Xmax plane.
			Secant method is used to solve the non linear Bspline equation 
			to get the parameter value for each co-ordinate point $x_{i}$.

            \item
            For each parameter value corresponding to $x_{i}$ a corresponding 
			$y_{i}$ is calculated from the profile equation.

            \item
            For each $i$, a plane normal to Y direction passing through $x_{i}$
            and a plane normal to the X direction passing through $y_{i}$ are
            contributed to the dimensioning procedure.

			\item
			At the dimension graph level, these contributed planes are used
			dimension planes in the dimension scheme which is dealt with in
            detail in section ~\ref{dimpts}.
			\end{itemize}

