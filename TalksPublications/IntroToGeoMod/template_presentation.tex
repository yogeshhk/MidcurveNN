\usepackage{beamerthemeshadow}
\usepackage{adjustbox}
\usepackage{lmodern}
\usepackage{booktabs}
%\usepackage{enumitem}
\usepackage[english]{babel}
\usepackage{tikz-qtree} 
%\usepackage{listings}
%\usepackage{flashmovie}
\usepackage{media9}
\usepackage{makecell}
 \usepackage{multimedia}
 \usepackage{listings}
 \usepackage{color}
 % \usepackage{helvet}
\definecolor{codegreen}{rgb}{0,0.6,0}
\definecolor{codegray}{rgb}{0.5,0.5,0.5}
\definecolor{codepurple}{rgb}{0.58,0,0.82}
\definecolor{backcolour}{rgb}{.914, .89, .957} % pale purple

\definecolor{mygreen}{rgb}{0,0.6,0}
\definecolor{mygray}{rgb}{0.5,0.5,0.5}
\definecolor{mymauve}{rgb}{0.58,0,0.82}

\lstdefinestyle{mystyle}{
  backgroundcolor=\color{backcolour},   % choose the background color; you must add \usepackage{color} or \usepackage{xcolor}; should come as last argument
  basicstyle=\footnotesize\ttfamily,       % the size of the fonts that are used for the code
  breakatwhitespace=true,          % sets if automatic breaks should only happen at whitespace
  breaklines=true,                 % sets automatic line breaking
  captionpos=b,                    % sets the caption-position to bottom
  commentstyle=\color{mygreen},    % comment style
  deletekeywords={...},            % if you want to delete keywords from the given language
  escapeinside={\%*}{*)},          % if you want to add LaTeX within your code
  extendedchars=true,              % lets you use non-ASCII characters; for 8-bits encodings only, does not work with UTF-8
  frame=single,	                   % adds a frame around the code
  keepspaces=true,                 % keeps spaces in text, useful for keeping indentation of code (possibly needs columns=flexible)
  keywordstyle=\color{blue},       % keyword style
  language=Python,                 % the language of the code
  morekeywords={*,...},            % if you want to add more keywords to the set
  numbers=left,                    % where to put the line-numbers; possible values are (none, left, right)
  numbersep=5pt,                   % how far the line-numbers are from the code
  numberstyle=\tiny\color{mygray}, % the style that is used for the line-numbers
  rulecolor=\color{black},         % if not set, the frame-color may be changed on line-breaks within not-black text (e.g. comments (green here))
  showspaces=false,                % show spaces everywhere adding particular underscores; it overrides 'showstringspaces'
  showstringspaces=false,          % underline spaces within strings only
  showtabs=false,                  % show tabs within strings adding particular underscores
  stepnumber=2,                    % the step between two line-numbers. If it's 1, each line will be numbered
  stringstyle=\color{mymauve},     % string literal style
  tabsize=2,	                   % sets default tabsize to 2 spaces
  columns=fullflexible,
  linewidth=0.98\linewidth,        % Box width
  aboveskip=10pt,	   			   % Space before listing 
  belowskip=-15pt,	   			   % Space after listing  
  xleftmargin=.02\linewidth,  
  title=\lstname                   % show the filename of files included with \lstinputlisting; also try caption instead of title
}


% \definecolor{codegreen}{rgb}{0,0.6,0}
% \definecolor{codegray}{rgb}{0.5,0.5,0.5}
% \definecolor{codepurple}{rgb}{0.58,0,0.82}
% %\definecolor{backcolour}{rgb}{0.95,0.95,0.92} % faint postman color
% \definecolor{backcolour}{rgb}{.914, .89, .957} % pale purple
% %\lstset{basicstyle=\footnotesize\ttfamily}

% \lstdefinestyle{mystyle}{
    % backgroundcolor=\color{backcolour},   
    % commentstyle=\color{codegreen},
    % keywordstyle=\color{magenta},
    % numberstyle=\tiny\color{codegray},
    % stringstyle=\color{codepurple},
    % basicstyle= \tiny\ttfamily %\scriptsize\ttfamily, %\footnotesize,  % the size of the fonts that are used for the code
    % breakatwhitespace=true,  % sets if automatic breaks should only happen at whitespace        
    % breaklines=true, % sets automatic line breaking   
    % linewidth=\linewidth,	
    % captionpos=b,                    
    % keepspaces=true,% keeps spaces in text, useful for keeping indentation                
% %    numbers=left,                  
    % numbers=none,  
% %    numbersep=5pt,                  
    % showspaces=false,                
    % showstringspaces=false,
    % showtabs=false,                  
    % tabsize=2
% }
\lstset{style=mystyle}

\usetheme{Warsaw}
\usecolortheme{whale}
\usefonttheme{structuresmallcapsserif}
\useinnertheme{rounded}
\useoutertheme[subsection=false,footline=authortitle]{miniframes}   

\setbeamerfont{block title}{size={}}

\definecolor{darkpurple}{rgb}{0.471,0.008,0.412}
\definecolor{middlepurple}{rgb}{0.663,0.012,0.580}
\definecolor{lightpurple}{rgb}{0.855,0.016,0.749}
\definecolor{gold}{RGB}{254,232,0}
\definecolor{nyupurple}{RGB}{90, 35, 140}
\definecolor{lightnyupurple}{RGB}{159, 120, 184}

\setbeamercolor{alerted text}{fg=nyupurple}
\setbeamercolor{example text}{fg=darkgray}

\setbeamercolor*{palette secondary}{fg=white,bg=lightnyupurple} %subsection
\setbeamercolor*{palette tertiary}{fg=white,bg=nyupurple} %section

\setbeamercolor{title}{fg=black, bg=lightnyupurple}         %Title of presentation
\setbeamercolor{block title}{fg=white, bg=nyupurple}
\setbeamercolor{section title}{fg=white, bg=nyupurple}
\setbeamercolor{subsection title}{fg=white, bg=lightnyupurple}
%\setbeamercolor{frametitle}{fg=white, bg=lightnyupurple}  % Gives gradient in the frame title bar
\setbeamertemplate{frametitle}[default][colsep=-4bp,rounded=false,shadow=false] % disable gradient

\setbeamerfont{frametitle}{size=\large}
\setbeamerfont{frametitle}{series=\bfseries}

\setbeamertemplate{blocks}[rounded][shadow=true] 
\setbeamertemplate{items}[triangle]
\setbeamercolor{item}{fg=gray, bg=white}

\setbeamertemplate{navigation symbols}{}
\setbeamertemplate{bibliography entry title}{}
\setbeamertemplate{bibliography entry location}{}
\setbeamertemplate{bibliography entry note}{}
\setbeamertemplate{bibliography item}{}

\beamertemplatenavigationsymbolsempty
\setbeamercovered{transparent}

\setbeamertemplate{itemize item}{\scriptsize\raise1.25pt\hbox{\donotcoloroutermaths$\blacktriangleright$}}
\setbeamertemplate{itemize subitem}{\tiny\raise1.5pt\hbox{\donotcoloroutermaths$\blacktriangleright$}}
\setbeamertemplate{itemize subsubitem}{\tiny\raise1.5pt\hbox{\donotcoloroutermaths$\blacktriangleright$}}
\setbeamertemplate{enumerate item}{\insertenumlabel.}
\setbeamertemplate{enumerate subitem}{\insertenumlabel.\insertsubenumlabel}
\setbeamertemplate{enumerate subsubitem}{\insertenumlabel.\insertsubenumlabel.\insertsubsubenumlabel}
\setbeamertemplate{enumerate mini template}{\insertenumlabel}
\setbeamertemplate{mini frames}{}
%\setbeamertemplate{itemize/enumerate subbody begin}{\vspace{1cm}}
%\setbeamertemplate{itemize/enumerate subbody end}{\vspace{1cm}}

% TOC at start of a Section or a Subsection
\AtBeginSection[] { 
  \begin{frame}[plain] 
    \frametitle{Agenda} 
    \tableofcontents[currentsection] 
  \end{frame} 
  \addtocounter{framenumber}{-1} 
} 

%\AtBeginSubsection[] { 
%  \begin{frame}[plain] 
%   \frametitle{Agenda} 
%    \tableofcontents[currentsubsection] 
%    \addtocounter{framenumber}{-1} 
%  \end{frame} 
%} 


%\logo{\includegraphics[width=2cm,keepaspectratio]{images/yati_logo.pdf}}
\author[Yogesh H Kulkarni]{Yogesh Kulkarni} 

\newcommand{\code}[1]{\par\vskip0pt plus 1filll \footnotesize Code:~\itshape#1}
