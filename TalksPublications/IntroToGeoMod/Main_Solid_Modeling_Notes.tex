\documentclass[12pt,a4paper,openbib]{article}
\usepackage{anysize,mflogo,xspace,texnames}
\usepackage{graphicx}

\newcommand{\teTeX}{\textrm{te}\TeX\xspace}
\newcommand{\Linux}{\textrm{Linu}\textsf{X}\xspace}
\newcommand{\smiley}{\texttt{ :-)}\xspace}

\title{ Introduction to Solid Modeling}
\author{Yogesh Kulkarni}
\date{February 1999}

\begin{document}
\maketitle
\section{Introduction}
\begin{itemize}
	\item 	{\bf What are models ?} - Model is an abstraction. It 
		is an artificially constructed object for better understanding,
		observation.
	\item	{\bf What are the types ?} -
		\begin{itemize}
			\item {\bf Physical Models} - share relative dimensions,
				and general appearance.
			\item {\bf Molecule Models} - share relative arrangement
				of molecules with few other properties.
			\item {\bf Mathematical Models} - represents behaviour
				in terms of numerical data and equations.
			\item {\bf Engineering drawings} - convey 3D info with
				2D representation.
			\item {\bf Computer Models} - use data stored in files.
		\end{itemize}
	\item {\bf Advantages} - Easier for study than the real counterpart as
		\begin{itemize}
			\item object may not exist, 
			\item may not be observable, 
			\item may not be observable in controlled fashion.
		\end{itemize}
\end{itemize}


\section{Geometric Modeling}
Its a type of Computer Model with aim to solve problems which are inherently
{\it geometric}. Here, we try to represent the geometric shape of the object 
adequately by the equations. {\em Solid Modeling} is a branch of {\em Geometric
Modeling} that emphasizes the general applicability of the models, and insists
on creating only ``complete'' representation of physical solid objects, i.e.
representation that are adequate for answering arbitrary geometric questions
{\it algorithmically} ( without interaction of user).

\section{Requirements of Solid representation}
\begin{itemize}
	\item {\bf Completeness} - Engineering drawing may not always get 
	interpreted into meaningful object. Even 3D wireframe models may have 
	several	interpretations in terms of solidity. Completeness demands 
	calculating answers to arbitrary geometric questions.
	\item {\bf Integrity} - 3D Models are constructed from planes rather than 
	lines. But this does not rule out possibility of self-intersecting 
	polyhedral models which are physically invalid. Integrity avoids 
	generation of incorrect models.
%            \begin{figure}[htp]
%                \begin{center}
%                    \includegraphics{Completeness.eps}
%                    \caption{A nonsense drawing}
%                \end{center}
%            \end{figure}

	\item {\bf Complexity} - Relatively simple object should have relatively 
	simpler representation. The difficulty of dealing with solid geometry 
	increases with complexity of mathematical formulation used.
	\item {\bf Computability} - Operations available in the Solid modeler
	should constitute a {\it closed system} where all the operations are
	guaranteed to maintain the correctness of the underlying model.e.g.
	Dual Entry System (debit-credit) in an account statement ensures balance
	at every entry.
\end{itemize}
\section{Mathematical Models of the Solids}
What is {\bf ``Solidity''} ? 
\\
We can have characterization based on two approaches, one stressing on the 
`` three dimensional solidity'' of things, and the other concentrating on the
boundedness by surfaces.
\subsection{Point-Set Models}
Consider 3D Euclidian \footnote{The one that can be represented by 3 numbers}
Space $E^{3}$. A solid is a bounded,closed subset of $E^{3}$.  This definition 
dose capture essence of ``Solidity'' but is far too large to be really useful 
in terms of computer storage.

\begin{itemize}
	\item {\bf Rigidity} -
		A rigid object is an equivalence class of point-set of $E^{3}$ spanned 			by following equivalance relation $\circ$ : Let $A,B$ be subsets of 
		$E^{3}$. Then $A \circ B$ holds iff $A$ can be mapped to $B$ with a 
		rigid transformation.
	\item {\bf Regularity} - We expect a solid to be ``all material''; it is 
		not impossible that a single point, or a line or even a 2D area would be
		missing from a solid. Likewise, a ``solid'' point-set must not contain
		``isolated points'',``isolated items'',or ``isolated faces'' that do not
		have any material around them.
%            \begin{figure}[htp]
%                \begin{center}
%                    \includegraphics{Regular.eps}
%                    \caption{A regular and a nonregular set}
%                \end{center}
%            \end{figure}

		The regularization of a point-set $A,r(A)$, is defined by
		\begin{equation}
			r(A) = c(i(A))
		\end{equation}
		where $c(A)$ and $i(A)$ denote the closure and interior of $A$. Sets
		that satisfy $r(A) \equiv A$ are said to be regular\footnote{This 
		tears off all ``isolated'' parts of a point set, covers it completely 
		with a tight skin and fills the result up with the material}
	\item {\bf Finiteness} - You can not store all the point-set data. Indirect
		but finite methods are adopted like a line is stored as two endpoints.
		Usually we must require that the surface of a solid is ``smooth''
		\footnote{Fractals are the exeptions as they are non-smooth planes
and still have indirect, recursive representation.}
\end{itemize}


\subsection{Surface based Models - Plane Models}
Its possible to characterize modeling spaces just based on properties of 
surfaces. The surface based characterization of the solids looks at the
boundary of a solid object. The boundary is considered to consist of a 
collection of ``faces'' that are ``glued'' together so that they form a
complete, closing ``skin'' around the object.
\begin{itemize}
	\item {\bf 2-Manifolds} - A surface may be regarded as a subset of
		$E^{3}$  which is essentially ``two-dimensional'': every point of
		the surface is surrounded by a ``two-dimensional'' region of points 
		belonging to the surface. A 2-manifold $M$ is a topological space where
		every point has a neighbourhood topologically equivalent to an open
		disk of $E^{2}$. Surface of earth is a 2-manifold.\\
		{\bf Limitations of Manifold models } - If a 2-manifold $M$ and the 
		boundary of $A$, $b(A)$, are topologically equivalent, we say that $A$ 
		is a {\bf realization} of $M$ in $E^{3}$. All objects are not 
		``realizable''.
		If the surface touches itself in a point or at a curve segment, the 
		neighbourhood of such points is not a simple disk but more than one
		disks. So such objects are considered as the rigid combination of two
		components that just happen to touch each other at a point.
%            \begin{figure}[htp]
%                \begin{center}
%                    \includegraphics{Realizability.eps}
%                    \caption{Solids with NonManifold surfaces}
%                \end{center}
%            \end{figure}

	\item {\bf Plane Model of 2-Manifold} - Let $P$ be a set of polygons, 
		and let $a_{1}, a_{2},...$ be a collection of edges of these polygons.  		These edges
		are termed ``identified''\footnote{common} when a new topology is
		defined on $P$ as follows:
		\begin{itemize}
			\item Each edge is assigned an orientation from one end point to the
				other with initial point at 0 and other point at 1.
			\item  The points on edges with same correspondig values are 
				considered as single points.
			\item The neighbourhood of $P$ has disks entirely contained in a
				single polygon plus the union of half disks diameters are 
				matching intervals around corresponding points on the edges.
		\end{itemize}
		A plane model is a planer directed graph $\{ N,A,R\}$ with finite 
		number of vertices $N = \{n_{1},n_{2},...\}$, edges $A = \{a_{1},a_{2},
		...\}$ and polygons $R= \{r_{1},r_{2},...\}$ bounded by edges and
		vertices. Each polygon has a certain orientation around its edges and
		vertices. Plane models are ``realizable'' if
%            \begin{figure}[htp]
%                \begin{center}
%                    \includegraphics{PlaneModel.eps}
%                    \caption{Plane models of a pyramid}
%                \end{center}
%            \end{figure}

		\begin{itemize}
			\item Every edge is identified with exactly one other edge.
			\item For each collection of identified vertices, the polygons
				identified at that collection can be arranged in cycle such 
				that each consecutive pair of polygons in the cycle is 
				identified at an edge adjacent to a vertex from the collection.
		\end{itemize}
		{\bf Orientability} - There are 2-Manifolds that do not have any
		physical counterpart in $E^{3}$ like klein bottle. These non-realizable
		2-Manifolds are distinguished from realizable ones by the concept
		of ``orientability''.
%            \begin{figure}[htp]
%                \begin{center}
%                    \includegraphics{Orientability.eps}
%                    \caption{Klien Bottle}
%                \end{center}
%            \end{figure}

		A plane model is ``orientable'' if the directions of its polygons can 
		be chosen so that for each pair of identical edges, one edge occurs in
		it positive orientation in the direction chosen for its polygon, and
		the other one in its negative direction. This condition is called
		{\bf M\"{o}bius' rule}.	
	\item {\bf Euler Characteristics} - Let the surface $S$ be given as a plane
		model and let $v,e,$ and $f$ denote the number of vertices, edges and
		faces in the plane model. Then the sum $ v - e + f$ is a constant
		independent of the manner in which $S$ is divided up to form the plane
		model. This constant is called ``Euler Characteristics'' of the
		surface and is denoted as $\chi(S)$. The theory of {\bf homology} gives
		\begin{equation}
			\chi = h_{0} - h_{1} + h_{2}
		\end{equation}
		where, $h_{0},h_{1},$ and $h_{2}$ are called the {\bf Betti numbers}.
		\begin{itemize}
			\item $h_{0}$ is number of connected pieces called ``componentes''.
					Hence of the cube. $h_{0} = 1$. Also called $s$ for 
					``shells''.\footnote{Addition of a loop on the face makes
					a new shell.}
			\item $h_{1}$ is largest number of closed curves that can be drawn
					without cutting the solid into two or more parts.
					Hence of the cube, $h_{1} = 0$ and doughnut has $h_{1} = 2$.
					$h_{1}/2$ denotes $h$, the number of holes.
			\item $h_{2}$ reveals the orientability of the surface and equals
					to $h_{0}$ for the kinds of surface we are looking for.
		\end{itemize}
		Finally, we get
		\begin{equation}
		 	v -e +f = 2(s -h)
		\end{equation}
%            \begin{figure}[htp]
%                \begin{center}
%                    \includegraphics[scale=0.7]{EulerCheck.eps}
%                    \caption{Checking Euler Equation with different cases}
%                \end{center}
%            \end{figure}

	\item {\bf Surfaces with Boundary} - Sometimes its necessary to have 
		surfaces that do not bound to get solid.  In case of Cylinder, some
		edges (boundary) are not ``identified'' to some other edge. These edges
		form a boundaries of the surface; hence the term ``Surfaces with 
		Boundary''. Theory of closed surfaces can be carried over to surfaces
		with boundary. To get cylinder, we have to cut two holes at corner
		of a elliptical shape and ``mend'' it to get cylinder. Its like
		``development'' in case of sheet metals. Euler characteristics of the
		mended surface is
		\begin{equation}
			\chi = v - e +f + b
		\end{equation}
		Hence for a surface with $b$ boundary components, the Euler-Poincar\'{e}
		formula can be written as
		\begin{equation}
		 	v -e +f = 2(s -h) - b
		\end{equation}
\end{itemize}
\section{Representation Schemes}
	\begin{itemize}
		\item {\bf Primitive Instancing} - Modeling using predefined libary 
			parts. For instance, we might model mechanical components by
			generating a sufficiantly large collection of ``Library'' parts.
			Models are simply instances of them, created with an
			instantiation operation. Instantiation requires actaul values 
			for parameters variables. Primitive Instancing is often used as
			an {\it auxiliary} scheme in modelers based on other
			representations in order to ease the description of often-needed
			parts e.g design of electrical instrumentaion or piping.
		\item {\bf Exhaustive Enumeration} - Solid is supposed to be composed
			of say tiny cubes which are nonoverlapping and uniform in size and
			orientaion, i.e. they form {\em regular subdivision} of space.
			Exhaustive enumeration is little used as it is extremely expensive 
			in memory terms, so space subdivision methods such as octree 
			methods are more frequently used as they can result in a huge 
			space saving at a small increase in algorithmic complexity. 
			Consider a component roughly 200mm long by 100mm wide by 200mm high.
		 	If we were required to model this part to an accuracy of only 0.1mm 			If we use a single bit for each voxel then, at eight (8) bits to 
			the byte we require approximately 500Mb of space, 
			so exhaustive enumeration is not compact.
			Exhaustive Enumeration is used in modeling special applications,
			such as modeling of the cell particles. 
		\item {\bf Octree Representaion} - Recursive subdivision of space of
			interest into eight {\em octants} that are arranged into 8-ary
			In this octant numbering system, octant 0 is the $x,y,z>0$ octant 
			and the octants are numbered anticlockwise around the $z$ axis,
			followed by those in the $z<0$ halfspace. The octree representation 			then given is equivalent to the object shown. The initial, top 
			level octants are centred on the origin. 
			If we took our object $200x200x100 mm^{3}$ then to model to an 
			accuracy of 1 part in 2000 (or 0.1mm) we would need an octree up 
			to 11 levels deep (212=2048). 
			However, we would make great space savings with all the areas that 
			were either entirely within the object or entirely without the 
			object.
			One particular application of octtrees is for pre-segmenting a B-rep 			model to allow rapid ray casting of the model by photorealistic 
			renderers. An octree is made of the B-rep model to some suitable 
			resolution and the nodes of the octree are tagged with those faces 
			of the model that can be found in the appropriate region. This 
			allows rapid ray-firing as precise intersections need only be 
			found for those nodes the ray passes through containing 
			interesting surfaces.
		\item {\bf Cell Decomposition} - Has a certain variety of basic cell
			types and a single operator {\em glue}. Indivisual cells are
			instances of parametrized cell types.
		\item {\bf Half Space Models} - To start from sufficiently simple
			point sets  that can be represented directly, and model other
			point sets in terms of very general combinations of the simple
			sets. A analytical function $f$ of $x,y,z$ can be defined in such
			a way that all points $X = (x,y,z), f(X) \geq 0$ are considered to
			belong to the point set, while $f(X) < 0$ defines its complement.
			Because $f(X) = 0$ divides the whole space into two subspaces, the
			two sets are defined as {\em half spaces}. Half Space models are
			constructed by combining instances of half-spaces with {\em 
			Bollean set operations} such as  union ($\cup$), intersection
			($\cap$) and set difference ($\setminus$).
%			\begin{figure}[htp]
%				\begin{center}
%					\includegraphics{HalfSpace.eps}
%					\caption{Half Space Model of a finite cylinder}
%				\end{center}
%			\end{figure}
		\item {\bf Constructive Solid Model (CSG)} - Its easier to operate on 
			bounded primitives than un-bounded half spaces. CSG modeler
			operates only on parametrized instances of {\em solid primitives}
			and boolean set operations on them.\\
				$<CSG tree>$ ::= 	
								$<primitive>$ $\mid$
								$<CSG tree><set operations><CSG tree>$ $\mid$ 
								$<CSG tree><rigid motion>$ \\
			The regularized set operations $union^{*}, intersection^{*}$, and
			$set difference^{*}$ , denoted by $\cup^{*},\cap^{*}$ and 
			$\setminus^{*}$ are defined as
			\begin{equation}
				A \cup^{*} B = c(i(A \cup B)) 
			\end{equation}
			\begin{equation}
				A \cap^{*} B = c(i(A \cap B)) 
			\end{equation}
			\begin{equation}
				A \setminus^{*} B = c(i(A \setminus B)) 
			\end{equation}
		\item {\bf Boundary representaion} - Represents solid indirectly 
			through a representation of its bounding surface.
	\end{itemize}
\section{Boundary Models}
Boundary models are based on the surface-oriented view to solid modeling. They
represent solid object by dividing its surface into a collection of {\em faces}.
A face may have several bounding curves provided they define a connected object.
In turn, the bounding curves of faces are represented through a division into
{\em edges} which in turn are chalked out in terms of {\em vertices}.
\subsection{Data Structure}
The basic three object types {\em face,edge} and {\em vertex}, along with the
constituent geometric information forms a boundary model. It also has 
information about how the faces, edges are related to each other, better known
as ``topology''. Many choices as to what information is stored explicitly and
what information is left implicit (i.e. to be computed as needed) must be made
when implementing a bounadry model.
\begin{itemize}
	\item {\bf Polygon-Based Boundary Models} - Have only planer faces. A solid
		consists of a collection of faces. {\em Graphical metafiles} have this
		kind of information.
	\item {\bf Vertex-Based Boundary Models} - Vertices are introduced as
		independant entries of the boundary data structure. Faces are 
		represented in terms og global vertex identifiers.
	\item {\bf Edge-Based Boundary Models} - If curved surfaces are present in
		a boundary model, it becomes useful to include also edge nodes 
		explicitly in the boundary data structure to be able to strore 
		information of intersection of curves of faces. Represents a face 
		boundary in terms of a closing sequence of edges, or {\em loop} for 
		short. Vertices of the faces are represented through edges.
	\item {\bf Winged-Edge Data Structure} - Its elaborated edge-based 
		geometry with introduction of {\em loop} information in edge nodes.
		Because each edge $e$ appears in exactly two faces, exactly two other
		edges $e\prime$ and $e\prime \prime$ appear after $e$ in these faces.
		So each edge has this new infomration of ``next'' edges, denoted by
		$ncw$ and $nccw$ for ``next clockwise'' and ``next counter clockwise'';
		in particular, $ncw$ identifies the next edge in the face where the 
		edge occurs in its positive orientation and $nccw$ the next edge in
		the other face. The full winged-edge data structure includes the
		identifiers $fcw$ and $fccw$ of its neighbour faces, and analogously
		to $ncw$ and $nccw$, the identifiers $pcw$ and $pccw$ for previous
		edges. To include faces with several boundaries (holes) each face has
		list of loops.
\end{itemize}
\subsection{Validity of Boundary Models} A boundary model is {\em valid} if it
defines the boundary of ``reasonable'' solid object. This includes following
conditions
\begin{itemize}
	\item The set of faces of the boundary model ``closes,'' i.e. forms the
		complete ``skin'' of the solid with no missing parts.(also called
		``topological integrity'').
	\item Faces of the model do not intersect each other except at the common
		vertices or the edges.
	\item The boundaries of the faces are simple polygons that do not intersect
		themselves.
\end{itemize}
	Unfortunately, the ``geometric integrity'' by above conditions can not be
	enforced just by structural means; by assigning inappropriate geometric
	data its possible to create invalid model with sound ``topological 
	integrity''.

\begin{appendix}
\section{Appendix}
The concept of nearness \footnote{Point $X$ is said to be near set $A$, if
every neighbourhood of $X$ contains $A$.} is fundamental in that it forms basis
of defining other topological concepts.
\begin{itemize}
\item $A$ set is {\bf closed} if it contains all its near points. The {\bf closure} of a set $A$ denoted by $c(A)$, is the set of points near $A$.
\item A set is {\bf bounded} if it is contained in a neighbourhood.
\item A set is {\bf compact} if it is closed and bounded.
\item A set is {\bf open} if no point of $A$ is near the complement of $A$.
\item The {\bf interior} of a set $A$, denoted by $i(A)$, is the set of points
	$\in A$ that are not near the complement of  $A$.
\item The {\bf boundary} of a set $A$, denoted by $b(A)$,is the set of points
that are near to both $A$ and its complement.
\end{itemize}
\end{appendix}

\begin{thebibliography}{6}


    \bibitem{Mantyala}
    Martti Mantyla.
    \textsl{Introduction to Solid Modeling}.
    Computer Science Press 1988,
    ISBN 07167-8015-1

\end{thebibliography}

\end{document}

