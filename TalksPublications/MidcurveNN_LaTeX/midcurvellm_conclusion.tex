%%%%%%%%%%%%%%%%%%%%%%%%%%%%%%%%%%%%%%%%%%%%%%%%%%%%%%%%%%%%%%%%%%%%%%%%%%%%%%%%%%
\begin{frame}[fragile]\frametitle{Summary}
	\begin{itemize}
	\item Traditional methods of computing midcurves are predominantly rules-based and thus, have limitation of not developing a generic model which will accept any input shape. 	
	\item A novel Large Language Model based approach attempts to build such a generic model. 
	\item One such model, GPT-4, seems to be very effective. Although other proprietary and open-source models need to catch-up with GPT-4, even GPT-4 needs to be developed further to understand not just sequential lines but graphs/networks with different shapes, essentially, the geometry.  
	\end{itemize}	
\end{frame}


%%%%%%%%%%%%%%%%%%%%%%%%%%%%%%%%%%%%%%%%%%%%%%%%%%%%%%%%%%%%%%%%%%%%%%%%%%%%%%%%%%
\begin{frame}[fragile]\frametitle{Summary}
    \begin{itemize}
        \item This research significantly advances midcurve computation by exploring the interplay between established methodologies and cutting-edge approaches, particularly integrating Large Language Models (LLMs).
        \item Emphasizing the nuanced nature of geometric dimension reduction, it identifies challenges in handling variable-length input data, representing intricate shapes, and addressing limitations in existing models.
    \end{itemize}
    

    \textbf{MidcurveLLM Architecture:}
    \begin{itemize}
        \item Utilizes an Encoder-Decoder architecture and B-rep structures.
        \item Showcases promise, despite discrepancies with ground truths.
    \end{itemize}
    
    
    \textbf{Implications and Future Directions:}
    \begin{itemize}
        \item Points to the need for more extensive datasets and refined training parameters.
        \item Serves as a crucial catalyst for advancing midcurve computation methodologies.
        \item Invites further scrutiny and advancements in the transformative intersection of geometry and advanced machine learning.
    \end{itemize}
\end{frame}
